\chapter{Forme chiuse su semplicemente connessi}
\pagestyle{plain}
\thispagestyle{empty}
\pagestyle{fancy}

La dimostrazione che ho riportato nel capitolo \ref{cap:integrali_curvilinei} riguardo all'esattezza delle forme chiuse richiede delle ipotesi molto forti. Ci chiediamo, adesso, se sia possibile indebolire ancora di
più le ipotesi affinché una $1-$forma differenziale chiusa sia anche esatta: in questa appendici cercheremo di rispondere a questa domanda, prefiggendoci il compito di dimostrare che le forme differenziali chiuse su insiemi semplicemente connessi sono tutte esatte. 
Questo risultato è estendibile agli insiemi contraibili e, naturalmente, questo risultato può essere generalizzato anche per le $k-$forme, tuttavia ci limiteremo al caso delle $1-$forme.

\section{Omotopia e gruppo fondamentale}

Sia $U \subseteq \mathbb{R}^n$ un insieme connesso. Abbiamo visto, nel capitolo \ref{cap:integrali_curvilinei}, che sono equivalenti le due affermazioni:
\begin{enumerate}[label=\protect\circled{\arabic*}] 
    \item $\omega$ è esatta;
    \item $\forall \gamma: [a, b] \to U, \gamma(a) = \gamma(b), \int_\gamma \omega = 0$.
\end{enumerate} 
Vogliamo utilizzare questo fatto per dimostrare che su insiemi semplicemente connessi, le forme chiuse sono esatte. Per fare ciò dobbiamo ragionare più in generale, introducendo alcuni aspetti dell'omotopia che
non avevamo trattato in precedenza. \\

Prima di procedere, dobbiamo innanzitutto andare nel più astratto: per generalizzare il risultato ottenuto dobbiamo fare riferimento a degli strumenti più astratti di quelli visti per adesso. Dobbiamo, tuttavia, procedere
"cautamente": per utilizzare correttamente questi strumenti è necessario che questi siano ben definiti e ben posti. Per fare questo, introduciamo inizialmente il concetto di spazio topologico, un concetto attorno a cui stiamo
"orbitando" da tutto il corso ma di cui non abbiamo avuto mai bisogno di utilizzare, siccome abbiamo sempre lavorato utilizzando la distanza euclidea. 

\begin{definition}[spazio topologico]
    Una coppia $(S, \tau)$ si dice spazio topologico se
    \begin{enumerate}[label=\protect\circled{\arabic*}]
        \item $S$ è un insieme;
        \item $\tau$ è una collezione di insiemi di $S$ tali che
        \begin{enumerate}
            \item $\emptyset, S \in \tau$; (\emph{l'insieme vuoto e } $X$ \emph{ appartengono a } $\tau$);
            \item $\forall Z \in \tau, \bigcup Z \in \tau$ (\emph{l'unione arbitraria di insiemi appartenenti a } $\tau$ \emph{ appartiene a } $\tau$);
            \item $\forall Z \in \tau, \bigcap Z \in \tau$ (\emph{l'intersezione finita di insiemi appartenenti a } $\tau$ \emph{ appartiene a } $\tau$).
        \end{enumerate}
    \end{enumerate}
    Diciamo che $\tau$ è una topologia su $X$. Inoltre diciamo che gli insiemi che appartengono a $\tau$ sono aperti in $X$. 
    \label{def:topological_space}
\end{definition}

Soffermiamoci un momento su questa definizione: la topologia, nonostante la definizione astratta, altro non è che uno spazio in cui abbiamo la nozione di apertura e chiusura. Possiamo infatti ben vedere come uno spazio metrico
altro non è che uno spazio topologico in cui la topologia è indotta dalla metrica: la definizione di insieme aperto e chiuso, infatti, sono state date tirando in causa la nozione di distanza; tuttavia questo non toglie la natura
topologica dello spazio. Anzi, più avanti mostreremo (per completezza) che una distanza induce proprio una topologia. \\
\begin{example}[esempio di spazio topologico]
    Consideriamo l'insieme $S = \{a,b,c \}$. Osserviamo che
    \begin{enumerate}[label=\protect\circled{\arabic*}]
        \item le collezioni $R = \{\emptyset, S, \{a \} \}$ e $X = \{ \emptyset, S, \{a\}, \{b\}, \{c\} \}$ altro non sono che topologie su $S$.
        \item la collezione $Z = \{\emptyset, S, \{ a \}, \{ b \} \}$ non è una topologia su $S$ siccome manca l'unione di $\{ a \}, \{ b \}$.
    \end{enumerate}
\end{example}

\begin{definition}[insieme chiuso]
    Sia $(S, \tau)$ uno spazio topologico, un sottoinsieme $A \subseteq S$ si dice chiuso se $S \setminus A$ è aperto.
    \begin{equation*}
        A \text{ è chiuso } \iff A^c \in \tau.
    \end{equation*}
\end{definition}
\begin{remark}
    Preso $(S, \tau)$ spazio topologico, allora $\emptyset, S$ sono sia aperti che chiusi in ogni topologia.
\end{remark}
Prima di procedere ulteriormente, vorrei innanzitutto far vedere che, preso $(X, d)$ spazio metrico, allora la distanza induce una topologia. Vogliamo mostrare il seguente teorema
\begin{theorem}[degli aperti metrici]
    Sia $(X, d)$ uno spazio metrico. Sia $\tau$ la famiglia dei sottoinsiemi $A \subseteq X$ tali che
    $$
    \forall a \in A, \exists \varepsilon > 0 : B(a, \varepsilon) \subseteq A.
    $$
    Allora $\tau$ è una topologia su $X$.
\end{theorem}
\begin{proof}
    La dimostrazione verte nel mostrare che $\tau$ soddisfa le buone proprietà richieste nella definizione \ref{def:topological_space}. Osserviamo che, chiaramnete, il vuoto e $X$ appartengono a $\tau$. Mostriamo
    adesso che $\tau$ è chiusa per unioni qualsiasi. Sia $\{ A_i \}_{i \in I} \subseteq \tau$: vogliamo far vedere che $\bigcup\limits_{i \in I} A_i \in \tau$. Allora, si osservi che preso $x \in \bigcup\limits_{i \in I} A_i \exists j \in I : x \in A_j$. Quindi sappiamo
    che $\exists \varepsilon > 0 : B(x, \varepsilon) \subseteq A_j$, ma siccome $A_j \subseteq \bigcup\limits_{i \in I} A_i \implies B(x, \varepsilon) \subseteq \bigcup\limits_{i \in I} A_i$.
    Passiamo adesso alle intersezioni finite. Sia $A = \bigcap\limits_{i=1}^n A_i$ con $A_i \in \tau \, \forall i \in \{ 1, \ldots, n\}$. Allora sappiamo che $\exists \varepsilon_1, \ldots, \varepsilon_n > 0: B(x, \varepsilon_i) \subseteq A_i$ con $i \in \{1, \ldots, n \}$. Ma allora si conclude necessariamente
    che, posto $\varepsilon = \min(\varepsilon_1, \ldots, \varepsilon_n)$, sappiamo che $\varepsilon > 0$ (siccome il minimo di $n$ numeri strettamente positivi) e $\forall i \in \{1, \ldots, n \}, B(x, \varepsilon) \subseteq B(x, \varepsilon_i) \subseteq A_i$ (siccome $\forall i \in \{1, \ldots, n \}, \varepsilon \leq \varepsilon_i$) $\implies B(x, \varepsilon) \subseteq \bigcap\limits_{i=1}^n A_i$.
\end{proof}

Abbiamo già definito il concetto di \emph{omotopia} nel capitolo \ref{cap:integrali_curvilinei}. La definizione, nel caso di spazi topologici, è la seguente:
\begin{definition}[omotopia]
    Siano $X, Y$ due spazi topologici e siano $f, g : X \to Y$ funzioni continue. Una omotopia di $f$ in $g$ è una funzione continua $H: X \times I \to Y$ tale che $\forall x \in X, H(x, 0) = f(x), H(x, 1) = g(x)$. In tal caso
    diremo che $f$ è omotopa a $g$ e si scrive $f \sim g$.
\end{definition}
Tuttavia in spazi topologici come si definisce la continuità? La prossima definizione risponderà alla nostra domanda
\begin{definition}[continuità in spazi topologici]
    Siano $X, Y$ due spazi topologici. Diremo che $f: X \to Y$ è continua se $\forall A \subseteq Y \text{ aperto}, f^{-1}(A) \text{ è aperto in } X$.
\end{definition}
\begin{remark}
    Confrontiamo questa definizione con il teorema \ref{thm:theo_c1}: nel caso di $\mathbb{R}^n$ o, in generale negli spazi metrici, si poteva dimostrare che valeva questo fatto. Ma in generale tale teorema è proprio il punto di partenza
    della topologia perché, in un certo senso, ci dice che per studiare la continuità non è necessario conoscere la distanza ma solamente la famiglia dei sottoinsiemi aperti: la topologia altro non è che la branca della matematica che si occupa
    di studiare le trasformazioni continue e le proprietà di uno spazio che sono conservate da una trasformazione continua.
\end{remark}
Prima di procedere nella dimostrazione di quanto prefisso, abbiamo bisogno di dimostrare una serie di proprietà dell'omotopia che necessitano, tuttavia, di qualche proprietà insiemistica e topologica che dobbiamo dimostrare. La prima fra tutti è la seguente identità
\begin{lemma}
    Siano $X, Y$ due insiemi e sia $f: X \to Y$ una funzione fra i due. Allora, presi due sottoinsiemi $A \subseteq X, B \subseteq Y$, vale la seguente identità
    \begin{equation*}
        f^{-1}(Y \setminus B) = X \setminus f^{-1}(B).
    \end{equation*}
    \label{lemma:preimage_diff}
\end{lemma}
\begin{proof}
    Per dimostrare questa affermazione è sufficiente mostrare, come al solito, che valgono le due inclusioni $\subseteq$ e $\supseteq$. \\
    $\boxed{\subseteq}$: sia $x \in f^{-1}(Y \setminus B)$, allora, per definizione, sappiamo che $f(x) \in Y \setminus B$. Ma allora sappiamo che $f(x) \in Y$ e $f(x) \not\in B$, da cui deduciamo allora che $x \in f^{-1}(Y) = X$ (banale) e $x \not\in f^{-1}(B)$ dunque $x \in X \setminus f^{-1}(B)$. \\
    $\boxed{\supseteq}$: sia $x \in X \setminus f^{-1}(B)$, allora sappiamo che $x \in X$ e $x \not\in f^{-1}(B)$. Ma allora $f(x) \in Y$ (banale) e $f(x) \not\in B$, dunque $f(x) \in Y \setminus B$, pertanto, per definizione, $x \in f^{-1}(Y \setminus B)$.
\end{proof}
\begin{lemma}
    Siano $X, Y$ due spazi topologici. Diremo che $f$ è continua se e solo se $\forall A \subseteq Y$ chiuso, $f^{-1}(A)$ è chiuso.
    \label{lemma:continuos_map_closed}
\end{lemma}
\begin{proof} \hspace{1cm} \\
    $\boxed{\Rightarrow}$: sappiamo che $f$ è continua se la preimmagine di aperti è aperta. Ma allora, preso $A \subseteq Y$, abbiamo che $f^{-1}(A)$ è aperto mentre
    $$
        X \setminus f^{-1}(A) = f^{-1}(Y \setminus A) = f^{-1}(A^c),
    $$
    dove la prima uguaglianza segue dalla definizione di complementare e dalla proprietà dimostrata nel lemma \ref{lemma:preimage_diff}. Sappiamo allora che $A$ è aperto $\implies A^c$ è chiuso e $f^{-1}(A)$ è aperto $\implies (f^{-1}(A))^c$ è chiuso. \\
    Dunque abbiamo che per ogni chiuso $A \subseteq Y$, abbiamo che $f^{-1}(A)$ è chiuso. \\
    $\boxed{\Leftarrow}$: sappiamo che $f^{-1}(A)$ è chiuso per ogni chiuso $A \subseteq Y$. Ma allora, preso $A \subseteq Y$ aperto, abbiamo che $f^{-1}(A^c)$ è chiuso, ma allora
    $$
        f^{-1}(A^c) = f^{-1}(Y \setminus A) = X \setminus f^{-1}(A)
    $$
    dove l'ultima uguaglianza segue sempre dal lemma \ref{lemma:preimage_diff}. Ma allora segue che $f^{-1}(A)$ è aperto, dunque $f$ è continua. 
\end{proof}
\begin{lemma}[dell'incollamento]
    Siano $X = A \cap B$, con $A$ e $B$ sottospazi topologici chiusi di $X$, allora $f: X \to Y$ è continua se e solo se $f_{|A}$ e $f_{|B}$ sono continue.
    \label{lemma:gluing_lemma}
\end{lemma}
\begin{proof} \hspace{1cm} \\
    $\boxed{\Rightarrow}$: banalmente, se $f$ è una funzione continua, allora avremo che $f_{|A}$ e $f_{|B}$ sono continue. Infatti se $C = f(A)$ e $D = f(B)$ sono chiusi, allora per ogni aperto $C' \subseteq C$ avremo che
    $f^{-1}(C')$ è aperto (per continuità di $f$) e similmente per ogni aperto di $D$. \\
    $\boxed{\Leftarrow}$:  sia adesso $C$ un insieme chiuso di $Y$ allora $f^{-1}(A) \cap A$ e $f^{-1}(B) \cap B$ sono, rispettivamente, due chiusi in $A$ e in $B$, essendo $f_{|A}$ e $f_{|B}$ continue e per il lemma \ref{lemma:continuos_map_closed}. Ma allora
    $f^{-1}(C) \cap A$ e $f^{-1}(C) \cap B$ sono chiusi in $X$, essendo $A$ e $B$ chiusi. Avremo allora che
    $$
    f^{-1}(C) = (f^{-1}(C) \cap A) \cup (f^{-1}(C) \cap B)
    $$
    è unione di chiusi, dunque è un chiuso. Allora $f$ è continua, sempre per il lemma \ref{lemma:continuos_map_closed}.
\end{proof}
\begin{prop}[l'omotopia è una relazione d'equivalenza]
    Siano $X, Y$ spazi topologici. Allora l'omotopia definisce una relazione d'equivalenza sull'insieme delle funzioni continue da $X$ in $Y$.
\end{prop}
\begin{proof} \hspace{1cm} \\
    Dobbiamo banalmente mostrare che valgono le proprietà di \emph{riflessività}, \emph{simmetria} e \emph{transitività} di cui gode un a relazione d'equivalenza. \\
    \emph{Riflessività}: ogni funzione è banalmente omotopa a sé stessa, possiamo infatti considerare $H(x, \lambda) = f(x)$. \\
    \emph{Simmetria}: sia $f \sim g$, allora osserviamo che $H(x, \lambda)$ è l'omotopia di $f$ in $g$ allora $H(x, 1-\lambda)$ è l'omotopia di $g$ in $f$. \\
    \emph{Transitività}: siano $f \sim g, g \sim z$, dove $f, g, z : X \to Y$ e consideriamo le relative omotopie $H$ di $f$ in $g$ e $K$ di $g$ in $z$. Allora possiamo definire l'omotopia
    $$
        Z: X \to Y = \begin{cases}
            H(x, 2\lambda) & \text{se } 0 \leq \lambda \leq \frac{1}{2} \\
            K(x, 2\lambda - 1) & \text{se } \frac{1}{2} \leq \lambda \leq 1
        \end{cases}
    $$
    e osservare che essa si tratta di un'omotopia di $f$ in $z$ siccome $Z(x, 0) = H(x, 0) = f(x)$ e $Z(x, 1) = K(x, 1) = z(x)$. La continuità dell'omotopia non è assolutamente banale, ma discende dal lemma \ref{lemma:gluing_lemma}.
\end{proof}
\begin{prop}
    Uno spazio topologico $X$ connesso per archi è semplicemente connesso se e solo se esiste un'unica classse di omotopia di cammini chiusi che collegano due punti di $X$.
\end{prop}
\begin{proof}
    Osserviamo che la connessione per archi in $X$ garantisce che $\forall x, y \in X, \exists \gamma: [0, 1] \to X : \gamma(0) = x, \gamma(1) = y$. \\
    $\boxed{\Rightarrow}$: assumiamo che $X$ sia connesso per archi. In tal caso abbiamo che, prese due curve $\gamma, \delta : [0, 1] \to X$ tali che $\gamma(0) = \delta(0) = x$ e $\gamma(1) = \delta(1) = y$, avremo che,
    presa $g : [0, 1] \to [0, 1]$ tale che $p \stackrel{g}{\mapsto} 1 - p$, allora $\gamma + g*\delta$ è un cammino chiuso. Ma sappiamo che $\pi_1(X, x_0) = 0$ (il gruppo fondamentale è banale) pertanto abbiamo che
    $$
    [\gamma] = [\gamma \bar{\delta} \delta] = [\delta],
    $$
    siccome il circuito $\gamma + g*\delta$ è un circuito banale su $x$. Concludiamo che esiste un'unica classe di omotopia di cammini chiusi. \\
    $\boxed{\Leftarrow}$: Se esiste un'unica classe di omotopia di cammini fra $x_0$ e $x_0$ allora segue banalmente che $\pi(X, x_0) = 0$. Dall'arbitrarietà di $x_0$ segue che lo spazio è semplicemente connesso. 
\end{proof}