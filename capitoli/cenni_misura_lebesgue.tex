\chapter{Cenni alla misura di Lebesgue}

In questo capitolo affronteremo il concetto di misura e daremo alcuni accenni all'impostazione teorica data da Lebesgue nello sviluppo
del suo concetto di misura. Alcuni approfondimenti inseriti sono stati presi da \cite{rudin}, tuttavia per una trattazione più accurata e completa rimando a \cite{ambrosio}.

\section{La misura}
Nel trattare insiemi a dimensione maggiore di $1$ siamo, ovviamente, interessati ad individuare un funzionale $m: \mathcal{P}(X) \to [0; +\infty]$ che riesca a generalizzare la nozione geometrica di \emph{area} che noi abbiamo a due dimensioni, ovvero che possieda le seguenti proprietà:
\begin{enumerate}[label=\protect\circled{\arabic*}]
	\item $m(\emptyset) = 0$
	\item Se $A, B \subseteq X$ tali che $A \subseteq B \implies m(A) \leq m(B)$
	\item Se $A, B \subseteq X$ tali che $A \cap B = \emptyset \implies m(A \cup B) = m(A) + m(B)$
\end{enumerate}