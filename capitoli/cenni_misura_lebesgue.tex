\chapter{Cenni alla misura di Lebesgue}

In questo capitolo affronteremo il concetto di misura e daremo alcuni accenni all'impostazione teorica data da Lebesgue nello sviluppo
del suo concetto di misura. Alcuni approfondimenti inseriti sono stati presi da \cite{rudin}, tuttavia per una trattazione più accurata e completa rimando a \cite{measure_theory}.

\section{La misura esterna}
Vogliamo adesso determinare una funzione di insiemi $m_n: \mathcal{P}(\mathbb{R}^n) \to [0; +\infty]$ che rappresenta il nostro concetto intuitivo di \emph{volume} dell'insieme, detto \emph{misura dell'insieme}. \\
L'intuizione ci porta a pensare che questa funzione goda delle seguenti proprietà:
\begin{enumerate}[label=\protect\circled{\arabic*}]
	\item $m_n(\emptyset) = 0$
	\item Se $A, B \subseteq X$ tali che $A \subseteq B \implies m_n(A) \leq m_n(B)$
	\item Se $A, B \subseteq X$ tali che $A \cap B = \emptyset \implies m_n(A \cup B) = m_n(A) + m_n(B)$
\end{enumerate}
\begin{definition}[misura di un intervallo e di un $n-$intervallo]
	Per ogni intervallo $I \subseteq \mathbb{R}$ limitato definiamo
	$$
	\mathit{l}(I) = \sup{I} - \inf{I}.
	$$
	Sia adesso $I_1 \times \ldots \times I_n \subseteq \mathbb{R}^n$ definiamo
	$$
		v_n \left( \varprod_{i=1}^n I_i \right) = \prod_{i=1}^n \mathit{l}(I_i) 
	$$
\end{definition}
\begin{definition}[misura di Lebesgue di un insieme $E$]
	Sia $E \subseteq \mathbb{R}^n$, definiamo la misura di Lebesgue di $E$ come
	\begin{equation}
		m_n(E) = \inf \left\{ \sum_{i=1}^{+\infty} v_n(J_i) : E \subseteq \bigcup_{i=1}^{+\infty} J_i \text{ e } J_i \subseteq \mathbb{R}^n \text{ è un } n\text{-intervallo} \right\}
		\label{eq:def_lebesgue}
	\end{equation}
\end{definition}
\begin{remark}
	Con un procedimento abbastanza simile a come abbiamo fatto per la misura di Peano-Jordan, approssimiamo un insieme $E$ tramite degli $n-$intervalli dall'esterno. La sostanziale differenza, per come stiamo introducendo la misura di Lebesgue, è che stiamo approssimando solamente "da fuori".
	In questa maniera possiamo misurare una classe di insiemi più ampia rispetto a quella della misura di Peano-Jordan, in cui si richiedeva che la misura approssimata "da dentro" coincideva con la misura approssimata "da fuori".
\end{remark}
Come vedremo fra un po', esistono degli insiemi che non sono misurabili per Lebesgue; dunque per ogni sottoinsieme di $X$ non è possibile definire una misura $m_n$ (non su tutti gli insiemi essa è additiva come richiesto, dunque non è una misura). \\ Ci basta, tuttavia, definire una misura \emph{esterna}, ovvero
una funzione che stima il volume di un qualunque insieme $E$ attraverso una collezione di insiemi che ricopre dall'esterno il nostro insieme. In questa maniera è possibile definire una funzione d'insieme definita per ogni insieme, tuttavia non rispetta la proprietà di additività numerabile che noi vogliamo per effettuare operazioni come i limiti: tramite 
il teorema di Caratheodory si può tuttavia mostrare che, presa una misura esterna $\mu^*$, esiste uno spazio di misura in cui la misura coincide con quella esterna. \\
Successivamente, dovremo cercare un criterio con cui stabilire se un insieme $E$ è misurabile o meno.
\begin{definition}
	Diremo che $\mu^*: \mathcal{P}(X) \mapsto [0; +\infty]$ è una misura esterna se
	\begin{enumerate}[label=\protect\circled{\arabic*}]
		\item $\mu^*(\emptyset) = 0$;
		\item $\mu^*(E) \leq \sum\limits_{i=1}^{+\infty} \mu^*(E_i)$ se $E \subseteq \bigcup\limits_{i=1}^{+\infty} E_i$
	\end{enumerate}
\end{definition}
\begin{prop}[monotonia della misura esterna]
	Se $\mu^*$ è una misura esterna allora
	$$
	A \subseteq B \implies \mu^*(A) \leq \mu^*(B)
	$$
	\label{prop:monotonia}
\end{prop}
\begin{proof}
	Consideriamo $E = A$ e il seguente insieme
	\begin{align*}
	E_j = \begin{cases} B & j=1 \\ \emptyset & j \geq 2	\end{cases}
	\end{align*}
	osserviamo allora che $E \subseteq \bigcup\limits_{i=1}^{+\infty} E_i$ allora
	$$
	\mu^*(E) = \mu^*(A) \stackrel{\text{per la \circled{2}}}{\leq} \sum_{j=1}^{+\infty} \mu^*(E_j) = \mu^*(E_1) = \mu^*(B)
	$$
	ovvero
	$$
	\mu^*(A) \leq \mu^*(B)
	$$
\end{proof}
\begin{prop}[subadditività finita]
	Le misure esterne sono finitamente subadditive, ovvero preso $E_1, \ldots, E_k$ avremo che 
	$$
	\mu^* \left( \bigcup_{i=1}^{k} E_i \right) \leq \sum_{i=1}^k \mu^*(E_i)
	$$
	\label{prop:sub_finita}
\end{prop}
\begin{proof}
	Siano $E_1, \ldots, E_k \subseteq \mathbb{R}^n$. Imponendo che $\forall j > k, E_j = \emptyset$, allora
	$$
	\forall E \subseteq \bigcup_{i=1}^{+\infty} E_i, \mu^*(E) \leq \sum_{i=1}^{+\infty} \mu^*(E_i) = \sum\limits_{i=1}^{k} \mu^*(E_i) + \sum\limits_{i=k+1}^{+\infty} \mu^*(\emptyset) = \sum\limits_{i=1}^k \mu^*(E_i) \implies \mu^*(E) \leq \sum_{i=1}^k \mu^*(E_i)
	$$
	ma allora, siccome abbiamo banalmente che $\bigcup\limits_{i=1}^{k} E_i \subseteq \bigcup\limits_{i=1}^k E_i$, allora
	$$
	\mu^* \left( \bigcup_{i=1}^k E_i \right) \leq \sum_{i=1}^k \mu^*(E_i)
	$$
\end{proof}
Vogliamo adesso mostrare che la funzione d'insieme $m_n$ da noi definita è effettivamente una misura esterna, dunque è un buon punto di partenza per poter definire una misura che possieda le proprietà che abbiamo enunciato all'inizio di questo capitolo, siccome il teorema di 
Caratheodory ci assicura di poter trovare uno spazio di misura in cui la misura esterna $\mu^*$ è effettivamente una misura (ovvero è additiva su ogni insieme dello sapzio e non subadditiva).
\begin{prop}[misura di Lebesgue è una misura esterna]
	$m_n: \mathcal{P}(X) \to [0; +\infty]$ è una misura esterna
	\label{prop:lebesgue_mis_esterna}
\end{prop}
\begin{proof}
	La dimostrazione consiste nel mostrare che $m_n: \mathcal{P}(X) \mapsto [0; +\infty]$ gode delle proprietà della misura esterna. \\
	La \circled{1} segue banalmente, siccome l'$n-$intervallo nullo è un ricoprimento dell'insieme nullo, dunque $m_n(\emptyset) = 0$.
	Per la \circled{2}, supponiamo naturalmente che $\sum\limits_{i=1}^{+\infty} m_n(E_i) < +\infty$ (altrimenti è banale) e prendiamo un insieme $E \subseteq \bigcup\limits_{i=1}^{+\infty} E_i$, osservando che $\forall i \in \mathbb{N}, E_i \subseteq \bigcup\limits_{k=1}^{+\infty} I_{k}^{(i)}$, pertanto
	$$
		E \subseteq \bigcup_{i=1}^{+\infty} \bigcup_{k=1}^{+\infty} I_k^{(i)}
	$$
	e questo vale per ogni ricoprimento di $\bigcup\limits_{i=1}^{+\infty} E_i$, dunque anche per quello che "fornisce" la misura di $\bigcup\limits_{i=1}^{+\infty} E_i$
	$$
		m_n(E) \leq \inf \left\{\sum_{i=1} l(E_i) \right\} = m_n \left( \bigcup_{i=1}^{+\infty} E_i \right)
	$$
	resta da mostrare che $m_n( \bigcup\limits_{i=1}^{+\infty} E_i ) \leq \sum\limits_{i=1}^{+\infty} m_n(E_i)$ . Questo può essere fatto fissando $\varepsilon > 0$ e osservando che $\forall i \in \mathbb{N}, \exists \bigcup\limits_{k=1}^{+\infty} I_{i}^{(k)} \supseteq E_i : \sum\limits_{k=1}^{+\infty} l(I_i^{(k)}) < m_n(E_i) + \frac{\varepsilon}{2^{i+1}}$, dunque
	$$
	m_n \left( \bigcup_{i=1}^{+\infty} E_i \right) \leq \sum_{i=1}^{+\infty} \sum_{k=1}^{+\infty} l(I_{i}^{(k)}) < \sum_{i=1}^{+\infty} (m_n(E_i) + \frac{\varepsilon}{2^{i+1}}) = \varepsilon + \sum_{i=1}^{+\infty} m_n(E_i)
	$$
	e, siccome $\varepsilon$ era arbitrario, otteniamo la tesi facendo il limite.
\end{proof}
\begin{definition}[insieme misurabile secondo Lebesgue]
	Diremo che $A \subseteq X$ è misurabile secondo Lebesgue se
	$$
	\forall S \subseteq X \implies m_n(S) = m_n(A \cap S) + m_n(S \setminus A)
	$$
	\label{def:mis_lebesgue}
\end{definition}
Per ogni insieme $X$ definiamo con $\mathfrak{M}(X)$ la classe di tutti gli insiemi misurabili secondo Lebesgue.
\begin{definition}[$\sigma-$algebra]
	Diremo che $\mathcal{F} \subseteq \mathcal{P}(X)$ è una $\sigma-$algebra se valgono
	\begin{enumerate}[label=\protect\circled{\arabic*}]
		\item $\emptyset, X \in \mathcal{F}$;
		\item $A \in \mathcal{F} \implies (X \setminus A) \in \mathcal{F}$
		\item $\{ A_j : j \geq 1 \} \subseteq \mathcal{F} \implies \bigcup\limits_{i=1}^{+\infty} A_i \in \mathcal{F}$
	\end{enumerate}
\end{definition}
\begin{definition}[misura]
	Diremo che $\mu: \mathcal{F} \mapsto [0; +\infty]$ è una misura se
	\begin{enumerate}[label=\protect\circled{\arabic*}]
		\item $\mathcal{F}$ è una $\sigma-$algebra;
		\item $\mu(\emptyset) = 0$;
		\item Se $( \{A_i : i \geq 1 \} \subseteq \mathcal{F}) \wedge (\forall i \neq j, A_i \cap A_j = \emptyset$) allora
		$$
			\mu \left( \bigcup_{i=1}^{+\infty} A_i \right) = \sum_{i=1}^{+\infty} \mu(A_i)
		$$
	\end{enumerate}
\end{definition}
\begin{theorem}[teorema di Caratheodory]
	La classe dei misurabili $\mathfrak{M}(X)$ è una $\sigma-$algebra e $m_n:\mathfrak{M}(X) \to [0; +\infty]$ è una misura
\end{theorem}
Diremo allora che $m_n: \mathfrak{M}(X) \to [0; +\infty]$ è la misura di Lebesgue su $X$. Per lo più noi lavoreremo supponendo che $X=\mathbb{R}^n$.
\begin{prop}
	La misura $m_n: \mathfrak{M}(X) \to [0; +\infty]$ è finitamente additiva
\end{prop}
\begin{proof}
La dimostrazione è equivalente a quella della proposizione~\ref{prop:sub_finita} sostituendo opportunamente alle minorazione le eguaglianze.
\end{proof}
\begin{remark}
Ammettendo l'assioma della scelta è possibile mostrare che esistono degli insiemi che non sono misurabili (come ampiamente detto in precedenza). L'esempio più famoso è
l'insieme di Vitali, che si può costruire dentro un qualunque insieme di misura positiva $E \in \mathfrak{M}(\mathbb{R}^n)$. Dalla definizione di misurabilità allora $\exists S_0 \subseteq \mathbb{R}^n$
tale che 
$$
m_n(S_0) \leq m_n(S_0 \cap E) + m_n(S_0 \setminus E)
$$
dunqque si perderebbe la $\sigma-$additività su insiemi disgiunti.
\end{remark}
\section{Insiemi trascurabili}
\begin{definition}[insiemi trascurabili]
	Diremo che $Z \subseteq \mathbb{R}^n$ ha misura nulla oppure che è trascurabile se $m_n(Z) = 0$
\end{definition}
\begin{exercise}
	Mostrare che un punto $x \in \mathbb{R}^n$ ha misura nulla
\end{exercise}
\begin{proof}[Svolgimento]
	Sia $x \in \mathbb{R}^n$ e consideriamo un $n-$intervallo $\varprod\limits_{i=1}^n I_i$ dove $I_i = (x_i - \frac{\varepsilon}{2}, x_i + \frac{\varepsilon}{2})$ ($x_i$ rappresenta la $i-$esima componente del punto $x$), dunque rappresenta un cubo di \emph{volume} pari a $\varepsilon^n$ per $i \in \{1, \ldots, n \}$ con $\varepsilon > 0$. Allora possiamo osservare che
	$$
	\{ x \} \subseteq \bigcup_{i=1}^{+\infty} J_i
	$$
	dove
	\begin{align*}
		J_i = \begin{cases}
			\varprod\limits_{i=1}^n (x_i - \frac{\varepsilon}{2}, x_i + \frac{\varepsilon}{2}) \\
			\emptyset
		\end{cases}
	\end{align*}
	ma allora
	$$
	m_n(\{ x \}) \subseteq \bigcup_{i=1}^{+\infty} v_n(J_i) = \varepsilon^n
	$$
	ma allora, siccome $\varepsilon > 0$ è un valore arbitrario, possiamo farne il limite per $\varepsilon \to 0$, dunque $m_n(\{ x \}) = 0$.
\end{proof}
\begin{prop}
	Se $Z \subseteq \mathbb{R}^n$ è numerabile, allora
	$$
		m_n(Z) = 0
	$$
\end{prop}
\begin{remark}
	Non tutti gli insiemi a misura nulla sono numerabili: un esempio è l'insieme di Cantor.
\end{remark}
\begin{proof}
	Se $Z$ è numerabile allora
	$$
		Z = \bigcup_{i=1}^{+\infty} \{ x_i \}
	$$
	dove $x_i: \mathbb{N} \to Z$ dunque
	$$
	m_n(Z) \leq \sum_{i=1}^{+\infty} m_n(\{ x_i \}) = 0 \implies m_n(Z) = 0
	$$
\end{proof}
\begin{cor}
	$\mathbb{Q}^n$ è un insieme trascurabile
\end{cor}
\begin{proof}
$\mathbb{Q}^n \cong \mathbb{N}$ dunque $m_n(\mathbb{Q}^n) = 0$
\end{proof}
\begin{cor}
	Gli insiemi a misura nulla sono misurabili per Lebesgue.
	\label{cor:mis_nulla_misur}
\end{cor}
\begin{proof}
In virtù della definizione~\ref{def:mis_lebesgue} dobbiamo verificare che, preso $Z$ insieme trascurabile, allora
$$
\forall S \subseteq \mathbb{R}^n, m_n(S) = m_n(S \setminus Z) + m_n(S \cap Z)
$$
Ora osserviamo che, siccome $S = (S \setminus Z) \cup (S \cap Z)$, avremo per la subadditività finita (di cui gode $m_n$ siccome nella proposizione~\ref{prop:lebesgue_mis_esterna} abbiamo mostrare che è una misura esterna)
che
$$
m_n(S) \leq m_n(S \setminus Z) \cup (S \cap Z)
$$
Per la disuguaglianza opposta osserviamo che
\begin{align*}
m_n(S) \geq m_n(S \setminus Z) \text{ e } m_n(Z) \geq m_n(S \cap Z)
\end{align*}
per monotonia della misura esterna (proposizione~\ref{prop:monotonia}), ma siccome $0=m_n(Z) \geq m_n(S \cap Z) \implies m_n(S \cap Z) = 0$ allora
$$
m_n(S) \geq m_n(S \setminus Z) = m_n(S \setminus Z) + m_n(S \cap Z) \implies m_n(S) \geq m_n(S \setminus Z) + m_n(S \cap Z)
$$
dunque $m_n(S) = m_n(S \setminus Z) + m_n(S \cap Z)$, ovvero la tesi.
\end{proof}
\begin{theorem}
	Tutti gli aperti di $\mathbb{R}^n$ sono misurabili secondo Lebesgue
\end{theorem}
\begin{cor}
	Tutti gli aperti di $\mathbb{R}^n$ sono misurabili secondo Lebesgue
\end{cor}
\begin{proof}[Dimostrazione (del corollario)]
	Siccome $m_n: \mathfrak{M}(\mathbb{R}^n) \to [0; +\infty]$ allora $\mathfrak{M}(\mathbb{R}^n)$ è una $\sigma-$algebra, dunque se $A \in \mathfrak{M}(\mathbb{R}^n) \implies A^c \in \mathfrak{M}(\mathbb{R}^n)$.
\end{proof}
\begin{cor}
	Tutte le unioni e le intersezioni numerabili di insiemi chiusi o aperti sono ancora misurabili secondo Lebesgue
\end{cor}
\begin{proof}
	Se $\{ B_i : i \geq 1 \} \subseteq \mathfrak{M}(\mathbb{R}^n) \implies \bigcup\limits_{i=1}^{+\infty} B_i \in \mathfrak{M}(\mathbb{R}^n)$ perché $\mathfrak{M}(\mathbb{R}^n)$ è una $\sigma-$algebra, mentre per le intersezioni numerabili abbiamo che se $B_i \in \mathfrak{M}(\mathbb{R}^n) \implies B_i^c \in \mathfrak{M}(\mathbb{R}^n)$ (sempre per le proprietà delle $\sigma-$algebre)
	e, per le leggi di De Morgan, avremo che $\left(\bigcup\limits_{i=1}^{+\infty} B_i^c \right)^c = \bigcap\limits_{i=1}^{+\infty} B_i \in \mathfrak{M}(\mathbb{R}^n)$
\end{proof}
\section{Funzioni misurabili secondo Lebesgue}
Andiamo adesso ad affrontare il concetto di funzione misurabile, il quale ci condurrà, naturalmente, verso il concetto di integrale di Lebesgue
\begin{definition}[funzioni misurabili secondo Lebesgue]
Sia $E \in \mathfrak{M}(\mathbb{R}^n)$ misurabile e $f: E \to \bar{\mathbb{R}}$. Diremo che $f$ è misurabile secondo Lebesgue se $\forall t \in \mathbb{R}$ l'insieme $f^{-1}([-\infty, t)) \in \mathfrak{M}(\mathbb{R}^n)$, ovvero esso è misurabile.
\end{definition}
\begin{remark}(\textbf{Topologia di $[-\infty, \infty]$})
	Ogni aperto di $[-\infty, +\infty]$ è un'unione arbitraria di intervalli che possono essere sia aperti oppure del tipo $[-\infty, t)$ o $(t, +\infty]$.
\end{remark}
\begin{prop}[caratterizzazione delle funzioni misurabili]
	$E \in \mathfrak{M}(\mathbb{R}^n), f: E \to \bar{\mathbb{R}}$ è misurabile se e solo se $\forall O \subseteq \bar{\mathbb{R}}$ aperto di $[-\infty, +\infty]$ abbiamo che $f^{-1}(O) \in M(\mathbb{R}^n)$
\end{prop}
\begin{prop}[le funzioni continue sono misurabili]
	Se $E \in \mathcal{R}^n, f: E \to \mathbb{R}$ è continua, allora è misurabile
\end{prop}
\begin{proof}
	Se $O \subseteq [-\infty, +\infty]$ è un aperto contenuto in $\mathbb{R}$ in realtà è un aperto di $\mathbb{R} \implies f^{-1}(O)$ è aperto in $E$ in virtù del teorema~\ref{thm:teo_cf1}, dunque avremo che $\exists \Omega \subseteq \mathbb{R}^n$ tale che
	$$
	f^{-1}(0) = E \cap \Omega \in \mathfrak{M}(\mathbb{R}^n) \implies f \text{ è misurabile }
	$$
	ovvero la tesi.
\end{proof}
\begin{prop}
	Se $f_n: E \to \bar{\mathbb{R}}$ è una successioni di funzioni misurabili e $f:E \to \bar{\mathbb{R}}$. Se $f_n \stackrel{n \to +\infty}{\to} f \, \, \forall x \in E$ allora $f$ è misurabile
\end{prop}
Prendiamo in causa una funzione che causava molti problemi all'integrale di Riemann, ovvero la funzione di Dirichlet. Al momento, consideriamo una sua \emph{variante} $\chi_k: [0, 1] \to \mathbb{R}$ definita come
\begin{align*}
	\chi_k(x) = \begin{cases}
		1 & \text{ se } x \in \{q_0, \ldots, q_k \} \\
		0 & \text{ se } x \not\in \{q_0, \ldots, q_k \}
	\end{cases}
\end{align*}
allora $\chi_k \stackrel{n \to +\infty}{\to} \chi$, ovvero la funzione di Dirichlet. Grazie alla precedente proposizione siamo in grado di concludere che $\chi$ è misurabile per Lebesgue, ma non per Riemann (sebbene $\forall k, \chi_k$ è Riemann-integrabile).
\begin{exercise}
Mostrare la misurabilità di $\chi$ tramite la definizione di funzione misurabile secondo Lebesgue.
\end{exercise}
\begin{proof}[Svolgimento]
	Osserviamo che $\chi: [0, 1] \mapsto \{0, 1\}$. Mostriamo il seguente lemma
	\begin{lemma}
		Sia $f: A \mapsto B$ e presi $B_1, B_2 \subseteq B$ allora
		$$
			f^{-1}(B_1 \cup B_2) = f^{-1}(B_1) \cup f^{-1}(B_2)
		$$
	\end{lemma}
	\begin{proof}
		Mostriamo che $f^{-1}(B_1 \cup B_2) \subseteq f^{-1}(B_1) \cup f^{-1}(B_2)$: per definizione sappiamo che $f^{-1}(B_1 \cup B_2) = \{x \in A: f(x) \in B_1 \cup B_2 \} \implies f(x) \in B_1 \vee f(x) \in B_2 \implies x \in f^{-1}(B_1) \vee x \in f^{-1}(B_2)$.
		Mostriamo che $f^{-1}(B_1 \cup B_2) \supseteq f^{-1}(B_1) \cup f^{-1}(B_2)$: supponiamo, per assurdo, che $\exists x \in f^{-1}(B_1) \cup f^{-1}(B_2) : x \not\in f^{-1}(B_1 \cup B_2) \implies f(x) \in B_1 \vee f(x) \in B_2$ e $f(x) \not\in B_1 \cup B_2$, il che è un assurdo.
	\end{proof}
	Dunque, in virtù di questo lemma appena mostrato, avremo che
	$$
		\chi^{-1}(\{0, 1\}) = \chi^{-1}(\{0 \}) \cup \chi^{-1}(\{1 \})
	$$
\end{proof}
e abbiamo che $\chi^{-1}({0}) = \mathbb{Q} \cap [0, 1]$ che è un'unione numerabile di punti, ovvero i numeri razionali all'interno di questo intervallo che, in virtù del corollario~\ref{cor:mis_nulla_misur}, è misurabile. \\
L'intervallo restante, invece, sarebbe pari a $[0, 1] \setminus (\mathbb{Q} \cap [0,1])$ che, tuttavia, è misurabile siccome $[0,1]$ è misurabile, $\mathbb{Q} \cap [0,1]$ è misurabile e le differenze sono misurabili secondo Lebesgue per le proprietà delle $\sigma-$algebre.
\begin{prop}
	Se $f, g: E \to \bar{\mathbb{R}}$ sono misurabili, allora
	\begin{enumerate}[label=\protect\circled{\arabic*}]
		\item se $f+g$ è ben definita ovunque su $E$ allora $f+g$ è misurabile;
		\item se $\lambda \in \mathbb{R}$ e $\lambda f$ è ben definita ovunque, allora $\lambda f$ è misurabile;
		\item $E \mapsto \max\{f(x), g(x) \}$ e $E \mapsto \min\{f(x),g(x) \}$ sono misurabili;
		\item se $f_j: E \to \mathbb{R}$ è misurabile $\forall j \in \mathbb{N} \implies x \in E \to \sup\limits_{j \in \mathbb{N}} f_j(x)$ è misurabile.
	\end{enumerate}
\end{prop}

