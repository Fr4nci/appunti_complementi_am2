\chapter{Il prodotto esterno}
Le forme differenziali, dal punto di vista storico, derivano dalla generalizzazione al caso multidimensionale di concetti
quali il lavoro di un campo su un cammino e il flusso di un liquido attraverso una superficie. In fisica classica, esse sono particolarmente
rilevanti per la piena comprensione del formalismo hamiltoniano della meccanica classica mentre in fisica moderna sono fondamentali per la
teoria della relatività generale e per la teoria dei campi quantistici.

\section{Forme algebriche esterne}
Vogliamo introdurre definendo le forme algebriche esterne, strettamente collegate alle forme differenziali. Partiamo dalle $1-$forme algebriche esterne
\begin{definition}
    Sia $\omega : \mathbb{R}^n \to \mathbb{R}$, diremo che essa è $1-$forma algebrica esterna se è una funzione lineare.
\end{definition}