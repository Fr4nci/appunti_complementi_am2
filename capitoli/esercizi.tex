\chapter{Altri esercizi proposti}
\begin{exercise}
	Sia $E=\{(x, y, z) \in \mathbb{R}^3 : z \geq 0, y \geq 1, y^{2 \alpha}(z + x^2) \leq 1 \}$. Determinare se è misurabile
\end{exercise}
\begin{proof}[Svolgimento]
	Osserviamo che la condizione $z + x^2 \leq y^{-2 \alpha} \implies 0 \leq y^{-2\alpha}$ e $x^2 \leq y^{-2 \alpha}$. Sezioniamo con Tonelli con la $y$
	$$
	\int_E dxdydz = \int_1^{+\infty} dy \int_{E_y} dxdz
	$$
	dove $E_y$ è un sezionamento a $y$ costante. Possiamo allora sezionare con $x^2 \leq y^{-2\alpha} \implies -y^{-\alpha} < x < y^{\alpha}$
	\begin{align*}
	&\int_1^{+\infty} dy \int_{E_y} dxdz = \int_1^{+\infty} dy \int_{-y^{-\alpha}}^{y^{-\alpha}} dx \int_0^{y^{-2 \alpha} - x^2} dz = 2 \int_1^{+\infty} dy \int_0^{y^{-\alpha}} y^{-2 \alpha} - x^2 dx = \\
	&=\int_1^{+\infty} y^{-3 \alpha} - \frac{x^3}{3}\Bigg|_0^{y^{-\alpha}} dy = \frac{4}{3} \frac{y^{-3\alpha + 1}}{1 - 3 \alpha}\Bigg|_1^{+\infty} \implies \alpha \neq \frac{1}{3}
	\end{align*}
\end{proof}
\begin{exercise}
Siano $\gamma(t)=(t^2 e^t, t^2)$ e siano $c(t)=(te, t)$ con $\gamma, c:[0,1] \to \mathbb{R}$. Calcolare l'area fra le due curve
\end{exercise}
\begin{proof}[Svolgimento]
	Possiamo usare la forma differenziale d'area per calcolare quest'area. \textbf{Attenzione:} usare una forma differenziale d'area rispetta ad un'altra fa cambiare il valore delle aree sotto le curve $\gamma$ e $c(t)$, in maniera tale che le differenze e le somme diano lo stesso valore: la forma differenziale
	$$
	\Gamma_c = \int_{\partial^{+} E} \frac{xdy - ydx}{2} = 0
	$$
	mentre
	$$
	\Gamma_c = \int_{\partial^{+} E} xdy = \frac{e}{2} \neq 0
	$$
	Quindi quando si usano le forme differenziali e si sceglie quale usare, dobbiamo fare i conti tutti con la stessa!
\end{proof}
\begin{exercise}
	Sia $\omega = \log(x^2 + y^2 + z^2)xdx + \log(x^2+y^2+z^2)ydy + \log(x^2+y^2+z^2)zdz$ definita su $\Omega = \mathbb{R}^3 \setminus \{(0, 0, 0)\}$. Mostrare che 
\end{exercise}