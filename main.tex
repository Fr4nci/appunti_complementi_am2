\documentclass[openany]{book}
%%%%%%%%%%%%%%%%%%%%%%%%%%%%%%%%%
% PACKAGE IMPORTS
%%%%%%%%%%%%%%%%%%%%%%%%%%%%%%%%%


\usepackage[tmargin=2cm,rmargin=1.5in,lmargin=1.5in,margin=0.85in,bmargin=2cm,footskip=.2in, a4paper]{geometry}
\usepackage{bookmark}
\usepackage{libertinus}
\usepackage{amsmath,amsfonts,amsthm,amssymb,mathtools}
\usepackage[varbb]{newpxmath}
%\usepackage[libertine]{newtxmath}
\usepackage{chemfig}
\usepackage{xfrac}
\usepackage{mhchem}
\usepackage[italian]{babel}
\usepackage[Sonny]{fncychap}
\usepackage[makeroom]{cancel}
\usepackage{mathtools}
\usepackage{listing}
\usepackage{enumitem}
\usepackage{hyperref,theoremref}
\hypersetup{
	pdftitle={Appunti di Complementi di Analisi Matematica 2},
	colorlinks=true, linkcolor=doc!90,
	bookmarksnumbered=true,
	bookmarksopen=true,
	pdftitle={\@title},
	pdfauthor={\@author}
}
\usepackage[most,many,breakable]{tcolorbox}
\usepackage{xcolor}
\usepackage{varwidth}
\usepackage{varwidth}
\usepackage{etoolbox}
%\usepackage{authblk}
\usepackage{nameref}
\usepackage{multicol,array}
\usepackage{tikz}
\usepackage{tikz-cd}
\usepackage{tikz-3dplot}
\usepackage{tikz}
\usetikzlibrary{3d,decorations.pathmorphing}
\usepackage[ruled,vlined,linesnumbered]{algorithm2e}
\usepackage{comment} % enables the use of multi-line comments (\ifx \fi) 
\usepackage{import}
\usepackage{xifthen}
\usepackage{pdfpages}
\usepackage{transparent}

\newcommand\mycommfont[1]{\footnotesize\ttfamily\textcolor{blue}{#1}}
\SetCommentSty{mycommfont}
\newcommand{\incfig}[1]{%
    \def\svgwidth{\columnwidth}
    \import{./figures/}{#1.pdf_tex}
}

\usepackage{tikzsymbols}
\usepackage{float}
\usepackage[toc, page]{appendix}
\renewcommand\qedsymbol{$\square$}
\usepackage{hyperref}
\usepackage{mwe}
\usepackage{fancyhdr}
\usepackage{pgfplots}
\usepackage{bm}
%\usepackage{import}
%\usepackage{xifthen}
%\usepackage{pdfpages}
%\usepackage{transparent}
\definecolor{doc}{RGB}{0,0,0}
\theoremstyle{definition}
\newtheorem{definition}{Definizione}[chapter]

\theoremstyle{definition}
\newtheorem{theorem}{Teorema}[chapter]
\theoremstyle{definition}
\newtheorem{cor}{Corollario}[theorem]
\theoremstyle{definition}
\newtheorem{lemma}{Lemma}[theorem]
\theoremstyle{plain}
\newtheorem{exercise}{Esercizio}[chapter]
\theoremstyle{definition}
\newtheorem{prop}{Proposizione}[chapter]

\theoremstyle{definition}
\newtheorem{example}{Esempio}[chapter]

\theoremstyle{remark}
\newtheorem*{remark}{Oss:}

\newcommand*\circled[1]{\tikz[baseline=(char.base)]{
		\node[shape=circle,draw,inner sep=1pt] (char) {#1};}}
\newcommand{\innerprod}[2]{\langle #1, #2 \rangle}
\renewcommand\appendixpagename{Appendici}
\renewcommand\appendixtocname{Appendici}
% Things Lie
\newcommand{\kb}{\mathfrak b}
\newcommand{\kg}{\mathfrak g}
\newcommand{\kh}{\mathfrak h}
\newcommand{\kn}{\mathfrak n}
\newcommand{\ku}{\mathfrak u}
\newcommand{\kz}{\mathfrak z}
\DeclareMathOperator{\Ext}{Ext} % Ext functor
\DeclareMathOperator{\Tor}{Tor} % Tor functor
\newcommand{\gl}{\opname{\mathfrak{gl}}} % frak gl group
\renewcommand{\sl}{\opname{\mathfrak{sl}}} % frak sl group chktex 6

% More script letters etc.
\newcommand{\SA}{\mathcal A}
\newcommand{\SB}{\mathcal B}
\newcommand{\SC}{\mathcal C}
\newcommand{\SF}{\mathcal F}
\newcommand{\SG}{\mathcal G}
\newcommand{\SH}{\mathcal H}
\newcommand{\OO}{\mathcal O}

\newcommand{\SCA}{\mathscr A}
\newcommand{\SCB}{\mathscr B}
\newcommand{\SCC}{\mathscr C}
\newcommand{\SCD}{\mathscr D}
\newcommand{\SCE}{\mathscr E}
\newcommand{\SCF}{\mathscr F}
\newcommand{\SCG}{\mathscr G}
\newcommand{\SCH}{\mathscr H}

% Mathfrak primes
\newcommand{\km}{\mathfrak m}
\newcommand{\kp}{\mathfrak p}
\newcommand{\kq}{\mathfrak q}

% number sets
\newcommand{\RR}[1][]{\ensuremath{\ifstrempty{#1}{\mathbb{R}}{\mathbb{R}^{#1}}}}
\newcommand{\NN}[1][]{\ensuremath{\ifstrempty{#1}{\mathbb{N}}{\mathbb{N}^{#1}}}}
\newcommand{\ZZ}[1][]{\ensuremath{\ifstrempty{#1}{\mathbb{Z}}{\mathbb{Z}^{#1}}}}
\newcommand{\QQ}[1][]{\ensuremath{\ifstrempty{#1}{\mathbb{Q}}{\mathbb{Q}^{#1}}}}
\newcommand{\CC}[1][]{\ensuremath{\ifstrempty{#1}{\mathbb{C}}{\mathbb{C}^{#1}}}}
\newcommand{\PP}[1][]{\ensuremath{\ifstrempty{#1}{\mathbb{P}}{\mathbb{P}^{#1}}}}
\newcommand{\HH}[1][]{\ensuremath{\ifstrempty{#1}{\mathbb{H}}{\mathbb{H}^{#1}}}}
\newcommand{\FF}[1][]{\ensuremath{\ifstrempty{#1}{\mathbb{F}}{\mathbb{F}^{#1}}}}
% expected value
\newcommand{\EE}{\ensuremath{\mathbb{E}}}
\newcommand{\charin}{\text{ char }}
\DeclareMathOperator{\sign}{sign}
\DeclareMathOperator{\Aut}{Aut}
\DeclareMathOperator{\Inn}{Inn}
\DeclareMathOperator{\Syl}{Syl}
\DeclareMathOperator{\Gal}{Gal}
\DeclareMathOperator{\GL}{GL} % General linear group
\DeclareMathOperator{\SL}{SL} % Special linear group

%---------------------------------------
% BlackBoard Math Fonts :-
%---------------------------------------

%Captital Letters
\newcommand{\bbA}{\mathbb{A}}	\newcommand{\bbB}{\mathbb{B}}
\newcommand{\bbC}{\mathbb{C}}	\newcommand{\bbD}{\mathbb{D}}
\newcommand{\bbE}{\mathbb{E}}	\newcommand{\bbF}{\mathbb{F}}
\newcommand{\bbG}{\mathbb{G}}	\newcommand{\bbH}{\mathbb{H}}
\newcommand{\bbI}{\mathbb{I}}	\newcommand{\bbJ}{\mathbb{J}}
\newcommand{\bbK}{\mathbb{K}}	\newcommand{\bbL}{\mathbb{L}}
\newcommand{\bbM}{\mathbb{M}}	\newcommand{\bbN}{\mathbb{N}}
\newcommand{\bbO}{\mathbb{O}}	\newcommand{\bbP}{\mathbb{P}}
\newcommand{\bbQ}{\mathbb{Q}}	\newcommand{\bbR}{\mathbb{R}}
\newcommand{\bbS}{\mathbb{S}}	\newcommand{\bbT}{\mathbb{T}}
\newcommand{\bbU}{\mathbb{U}}	\newcommand{\bbV}{\mathbb{V}}
\newcommand{\bbW}{\mathbb{W}}	\newcommand{\bbX}{\mathbb{X}}
\newcommand{\bbY}{\mathbb{Y}}	\newcommand{\bbZ}{\mathbb{Z}}

%---------------------------------------
% MathCal Fonts :-
%---------------------------------------

%Captital Letters
\newcommand{\mcA}{\mathcal{A}}	\newcommand{\mcB}{\mathcal{B}}
\newcommand{\mcC}{\mathcal{C}}	\newcommand{\mcD}{\mathcal{D}}
\newcommand{\mcE}{\mathcal{E}}	\newcommand{\mcF}{\mathcal{F}}
\newcommand{\mcG}{\mathcal{G}}	\newcommand{\mcH}{\mathcal{H}}
\newcommand{\mcI}{\mathcal{I}}	\newcommand{\mcJ}{\mathcal{J}}
\newcommand{\mcK}{\mathcal{K}}	\newcommand{\mcL}{\mathcal{L}}
\newcommand{\mcM}{\mathcal{M}}	\newcommand{\mcN}{\mathcal{N}}
\newcommand{\mcO}{\mathcal{O}}	\newcommand{\mcP}{\mathcal{P}}
\newcommand{\mcQ}{\mathcal{Q}}	\newcommand{\mcR}{\mathcal{R}}
\newcommand{\mcS}{\mathcal{S}}	\newcommand{\mcT}{\mathcal{T}}
\newcommand{\mcU}{\mathcal{U}}	\newcommand{\mcV}{\mathcal{V}}
\newcommand{\mcW}{\mathcal{W}}	\newcommand{\mcX}{\mathcal{X}}
\newcommand{\mcY}{\mathcal{Y}}	\newcommand{\mcZ}{\mathcal{Z}}


%---------------------------------------
% Bold Math Fonts :-
%---------------------------------------

%Captital Letters
\newcommand{\bmA}{\boldsymbol{A}}	\newcommand{\bmB}{\boldsymbol{B}}
\newcommand{\bmC}{\boldsymbol{C}}	\newcommand{\bmD}{\boldsymbol{D}}
\newcommand{\bmE}{\boldsymbol{E}}	\newcommand{\bmF}{\boldsymbol{F}}
\newcommand{\bmG}{\boldsymbol{G}}	\newcommand{\bmH}{\boldsymbol{H}}
\newcommand{\bmI}{\boldsymbol{I}}	\newcommand{\bmJ}{\boldsymbol{J}}
\newcommand{\bmK}{\boldsymbol{K}}	\newcommand{\bmL}{\boldsymbol{L}}
\newcommand{\bmM}{\boldsymbol{M}}	\newcommand{\bmN}{\boldsymbol{N}}
\newcommand{\bmO}{\boldsymbol{O}}	\newcommand{\bmP}{\boldsymbol{P}}
\newcommand{\bmQ}{\boldsymbol{Q}}	\newcommand{\bmR}{\boldsymbol{R}}
\newcommand{\bmS}{\boldsymbol{S}}	\newcommand{\bmT}{\boldsymbol{T}}
\newcommand{\bmU}{\boldsymbol{U}}	\newcommand{\bmV}{\boldsymbol{V}}
\newcommand{\bmW}{\boldsymbol{W}}	\newcommand{\bmX}{\boldsymbol{X}}
\newcommand{\bmY}{\boldsymbol{Y}}	\newcommand{\bmZ}{\boldsymbol{Z}}
%Small Letters
\newcommand{\bma}{\boldsymbol{a}}	\newcommand{\bmb}{\boldsymbol{b}}
\newcommand{\bmc}{\boldsymbol{c}}	\newcommand{\bmd}{\boldsymbol{d}}
\newcommand{\bme}{\boldsymbol{e}}	\newcommand{\bmf}{\boldsymbol{f}}
\newcommand{\bmg}{\boldsymbol{g}}	\newcommand{\bmh}{\boldsymbol{h}}
\newcommand{\bmi}{\boldsymbol{i}}	\newcommand{\bmj}{\boldsymbol{j}}
\newcommand{\bmk}{\boldsymbol{k}}	\newcommand{\bml}{\boldsymbol{l}}
\newcommand{\bmm}{\boldsymbol{m}}	\newcommand{\bmn}{\boldsymbol{n}}
\newcommand{\bmo}{\boldsymbol{o}}	\newcommand{\bmp}{\boldsymbol{p}}
\newcommand{\bmq}{\boldsymbol{q}}	\newcommand{\bmr}{\boldsymbol{r}}
\newcommand{\bms}{\boldsymbol{s}}	\newcommand{\bmt}{\boldsymbol{t}}
\newcommand{\bmu}{\boldsymbol{u}}	\newcommand{\bmv}{\boldsymbol{v}}
\newcommand{\bmw}{\boldsymbol{w}}	\newcommand{\bmx}{\boldsymbol{x}}
\newcommand{\bmy}{\boldsymbol{y}}	\newcommand{\bmz}{\boldsymbol{z}}

%---------------------------------------
% Scr Math Fonts :-
%---------------------------------------

\newcommand{\sA}{{\mathscr{A}}}   \newcommand{\sB}{{\mathscr{B}}}
\newcommand{\sC}{{\mathscr{C}}}   \newcommand{\sD}{{\mathscr{D}}}
\newcommand{\sE}{{\mathscr{E}}}   \newcommand{\sF}{{\mathscr{F}}}
\newcommand{\sG}{{\mathscr{G}}}   \newcommand{\sH}{{\mathscr{H}}}
\newcommand{\sI}{{\mathscr{I}}}   \newcommand{\sJ}{{\mathscr{J}}}
\newcommand{\sK}{{\mathscr{K}}}   \newcommand{\sL}{{\mathscr{L}}}
\newcommand{\sM}{{\mathscr{M}}}   \newcommand{\sN}{{\mathscr{N}}}
\newcommand{\sO}{{\mathscr{O}}}   \newcommand{\sP}{{\mathscr{P}}}
\newcommand{\sQ}{{\mathscr{Q}}}   \newcommand{\sR}{{\mathscr{R}}}
\newcommand{\sS}{{\mathscr{S}}}   \newcommand{\sT}{{\mathscr{T}}}
\newcommand{\sU}{{\mathscr{U}}}   \newcommand{\sV}{{\mathscr{V}}}
\newcommand{\sW}{{\mathscr{W}}}   \newcommand{\sX}{{\mathscr{X}}}
\newcommand{\sY}{{\mathscr{Y}}}   \newcommand{\sZ}{{\mathscr{Z}}}


%---------------------------------------
% Math Fraktur Font
%---------------------------------------

%Captital Letters
\newcommand{\mfA}{\mathfrak{A}}	\newcommand{\mfB}{\mathfrak{B}}
\newcommand{\mfC}{\mathfrak{C}}	\newcommand{\mfD}{\mathfrak{D}}
\newcommand{\mfE}{\mathfrak{E}}	\newcommand{\mfF}{\mathfrak{F}}
\newcommand{\mfG}{\mathfrak{G}}	\newcommand{\mfH}{\mathfrak{H}}
\newcommand{\mfI}{\mathfrak{I}}	\newcommand{\mfJ}{\mathfrak{J}}
\newcommand{\mfK}{\mathfrak{K}}	\newcommand{\mfL}{\mathfrak{L}}
\newcommand{\mfM}{\mathfrak{M}}	\newcommand{\mfN}{\mathfrak{N}}
\newcommand{\mfO}{\mathfrak{O}}	\newcommand{\mfP}{\mathfrak{P}}
\newcommand{\mfQ}{\mathfrak{Q}}	\newcommand{\mfR}{\mathfrak{R}}
\newcommand{\mfS}{\mathfrak{S}}	\newcommand{\mfT}{\mathfrak{T}}
\newcommand{\mfU}{\mathfrak{U}}	\newcommand{\mfV}{\mathfrak{V}}
\newcommand{\mfW}{\mathfrak{W}}	\newcommand{\mfX}{\mathfrak{X}}
\newcommand{\mfY}{\mathfrak{Y}}	\newcommand{\mfZ}{\mathfrak{Z}}
%Small Letters
\newcommand{\mfa}{\mathfrak{a}}	\newcommand{\mfb}{\mathfrak{b}}
\newcommand{\mfc}{\mathfrak{c}}	\newcommand{\mfd}{\mathfrak{d}}
\newcommand{\mfe}{\mathfrak{e}}	\newcommand{\mff}{\mathfrak{f}}
\newcommand{\mfg}{\mathfrak{g}}	\newcommand{\mfh}{\mathfrak{h}}
\newcommand{\mfi}{\mathfrak{i}}	\newcommand{\mfj}{\mathfrak{j}}
\newcommand{\mfk}{\mathfrak{k}}	\newcommand{\mfl}{\mathfrak{l}}
\newcommand{\mfm}{\mathfrak{m}}	\newcommand{\mfn}{\mathfrak{n}}
\newcommand{\mfo}{\mathfrak{o}}	\newcommand{\mfp}{\mathfrak{p}}
\newcommand{\mfq}{\mathfrak{q}}	\newcommand{\mfr}{\mathfrak{r}}
\newcommand{\mfs}{\mathfrak{s}}	\newcommand{\mft}{\mathfrak{t}}
\newcommand{\mfu}{\mathfrak{u}}	\newcommand{\mfv}{\mathfrak{v}}
\newcommand{\mfw}{\mathfrak{w}}	\newcommand{\mfx}{\mathfrak{x}}
\newcommand{\mfy}{\mathfrak{y}}	\newcommand{\mfz}{\mathfrak{z}}

%From M275 "Topology" at SJSU
\newcommand{\id}{\mathrm{id}}
\newcommand{\taking}[1]{\xrightarrow{#1}}
\newcommand{\inv}{^{-1}}

%From M170 "Introduction to Graph Theory" at SJSU
\DeclareMathOperator{\diam}{diam}
\DeclareMathOperator{\ord}{ord}
\newcommand{\defeq}{\overset{\mathrm{def}}{=}}

%From the USAMO .tex files
\newcommand{\ts}{\textsuperscript}
\newcommand{\dg}{^\circ}
\newcommand{\ii}{\item}

% % From Math 55 and Math 145 at Harvard
% \newenvironment{subproof}[1][Proof]{%
% \begin{proof}[#1] \renewcommand{\qedsymbol}{$\blacksquare$}}%
% {\end{proof}}

\newcommand{\liff}{\leftrightarrow}
\newcommand{\lthen}{\rightarrow}
\newcommand{\opname}{\operatorname}
\newcommand{\surjto}{\twoheadrightarrow}
\newcommand{\injto}{\hookrightarrow}
\newcommand{\On}{\mathrm{On}} % ordinals
\DeclareMathOperator{\img}{im} % Image
\DeclareMathOperator{\Img}{Im} % Image
\DeclareMathOperator{\coker}{coker} % Cokernel
\DeclareMathOperator{\Coker}{Coker} % Cokernel
\DeclareMathOperator{\Ker}{Ker} % Kernel
\DeclareMathOperator{\rank}{rank}
\DeclareMathOperator{\Spec}{Spec} % spectrum
\DeclareMathOperator{\Tr}{Tr} % trace
\DeclareMathOperator{\pr}{pr} % projection
\DeclareMathOperator{\ext}{ext} % extension
\DeclareMathOperator{\pred}{pred} % predecessor
\DeclareMathOperator{\dom}{dom} % domain
\DeclareMathOperator{\ran}{ran} % range
\DeclareMathOperator{\Hom}{Hom} % homomorphism
\DeclareMathOperator{\Mor}{Mor} % morphisms
\DeclareMathOperator{\End}{End} % endomorphism

\newcommand{\eps}{\epsilon}
\newcommand{\veps}{\varepsilon}
\newcommand{\ol}{\overline}
\newcommand{\ul}{\underline}
\newcommand{\wt}{\widetilde}
\newcommand{\wh}{\widehat}
\newcommand{\vocab}[1]{\textbf{\color{blue} #1}}
\providecommand{\half}{\frac{1}{2}}
\newcommand{\dang}{\measuredangle} %% Directed angle
\newcommand{\ray}[1]{\overrightarrow{#1}}
\newcommand{\seg}[1]{\overline{#1}}
\newcommand{\arc}[1]{\wideparen{#1}}
\DeclareMathOperator{\cis}{cis}
\DeclareMathOperator*{\lcm}{lcm}
\DeclareMathOperator*{\argmin}{arg min}
\DeclareMathOperator*{\argmax}{arg max}
\newcommand{\cycsum}{\sum_{\mathrm{cyc}}}
\newcommand{\symsum}{\sum_{\mathrm{sym}}}
\newcommand{\cycprod}{\prod_{\mathrm{cyc}}}
\newcommand{\symprod}{\prod_{\mathrm{sym}}}
\newcommand{\Qed}{\begin{flushright}\qed\end{flushright}}
\newcommand{\parinn}{\setlength{\parindent}{1cm}}
\newcommand{\parinf}{\setlength{\parindent}{0cm}}
% \newcommand{\norm}{\|\cdot\|}
\newcommand{\inorm}{\norm_{\infty}}
\newcommand{\opensets}{\{V_{\alpha}\}_{\alpha\in I}}
\newcommand{\oset}{V_{\alpha}}
\newcommand{\opset}[1]{V_{\alpha_{#1}}}
\newcommand{\lub}{\text{lub}}
\newcommand{\del}[2]{\frac{\partial #1}{\partial #2}}
\newcommand{\Del}[3]{\frac{\partial^{#1} #2}{\partial^{#1} #3}}
\newcommand{\deld}[2]{\dfrac{\partial #1}{\partial #2}}
\newcommand{\Deld}[3]{\dfrac{\partial^{#1} #2}{\partial^{#1} #3}}
\newcommand{\lm}{\lambda}
\newcommand{\uin}{\mathbin{\rotatebox[origin=c]{90}{$\in$}}}
\newcommand{\usubset}{\mathbin{\rotatebox[origin=c]{90}{$\subset$}}}
\newcommand{\lt}{\left}
\newcommand{\rt}{\right}
\newcommand{\bs}[1]{\boldsymbol{#1}}
\newcommand{\exs}{\exists}
\newcommand{\st}{\strut}
\newcommand{\dps}[1]{\displaystyle{#1}}

\newcommand{\sol}{\setlength{\parindent}{0cm}\textbf{\textit{Soluzione:}}\setlength{\parindent}{1cm} }
\newcommand{\solve}[1]{\setlength{\parindent}{0cm}\textbf{\textit{Soluzione: }}\setlength{\parindent}{1cm}#1 \Qed}
\DeclareMathOperator{\Div}{div}
\DeclareMathOperator{\Rot}{rot}
\newcommand{\dcdot}{%
  \mathbin{%
    \nonscript\mspace{-\muexpr\medmuskip*2/3}%
    \cdot
    \nonscript\mspace{-\muexpr\medmuskip*2/3}%
  }%
}

\usepackage{fancyhdr}
\usepackage{titlesec}  % Opzionale, per controllo avanzato sui titoli

\pagestyle{fancy}

% Definizione dell'intestazione per pagine dispari (Odd)
\fancyhead[LO]{\textbf{\thepage}}  % Numero di pagina in grassetto a sinistra
\fancyhead[RO]{\rightmark}         % Capitolo e sezione a destra

% Definizione dell'intestazione per pagine pari (Even)
\fancyhead[LE]{\leftmark}          % Capitolo e sezione a sinistra
\fancyhead[RE]{\textbf{\thepage}}  % Numero di pagina in grassetto a destra

% Cancella il footer
\fancyfoot{}

% Personalizzazione della linea orizzontale
\renewcommand{\headrulewidth}{0.4pt}
\renewcommand{\footrulewidth}{0pt}

% Personalizza il comando per segnare il capitolo
\renewcommand{\chaptermark}[1]{%
  \markboth{\thechapter.\ #1}{}}

% Personalizza il comando per segnare la sezione (include il capitolo)
\renewcommand{\sectionmark}[1]{%
  \markright{\thesection\ #1}}
\title{Appunti di Complementi di Analisi Matematica 2}
\author{Francesco Sermi}
\date{\today}

\begin{document}
	\begin{titlepage}
	\centering
	\vspace*{3cm}
	{\huge\bfseries Appunti di Complementi di AM 2 \par}
	\vspace{2cm}
	{\Large\itshape Francesco Sermi\par}
	\vfill
	{\large \hfill Pisa, \today \par}
	\end{titlepage}
	\tableofcontents
	\chapter*{Notazioni}
	Riporto all'inizio del libro le notazioni adottate all'interno di questo documento:
	\begin{itemize}[label=\hspace{-0.5em}]
		\item $\lim$ limite		
		\item $\frac{\partial f}{\partial x_i}, \partial_{e_i} f, \partial_{x_i} f, f_{x_i}$ derivata parziale rispetto a $x_i$
		\item $\frac{\partial f}{\partial v}, \partial_v f$ derivata direzionale rispetto a $v$
		\item $\nabla$ gradiente
		\item $\nabla \cdot A$ divergenza del campo vettoriale $A$
		\item $\nabla \times A$ rotore del campo vettoriale $A$
		\item $\underline{v}$ vettore (alcune volte senza il trattino)
		\item $||\cdot||, |\cdot|$ norma
		\item $\innerprod{\cdot}{\cdot}$ prodotto scalare/hermitiano
		\item $a \times b, a \wedge b$ prodotto vettoriale
		\item $\text{Int}(A)$ parte interna di A
		\item $\bar{A}$ chiusura di A
		\item $\partial A$ derivato di A
		\item $B(x_0, r)$ palla aperta di raggio $r$ centrata in $x_0$
		\item $\mathbb{B}(x_0, r)$ palla chiusa di raggio $r$ centrata in $x_0$
		\item $o(f)$ o-piccolo di $f$
		\item $O(f)$ O-grande di $f$
	\end{itemize}
	\chapter{Spazi euclidei}
	In questo capitolo, ci soffermeremo su alcuni risultati che possono essere ottenuti considerando uno spazio euclideo e sullo studio della sua topologia, siccome molti concetti dell'Analisi 1 devono essere concettualmente rivisti per poter essere generalizzati a spazi di dimensione diversa da 1.
	\section{Spazio euclideo e prodotto scalare}
	Partiamo ricordando al lettore la definizione di prodotto scalare/hermitiano e di \textbf{spazio euclideo}:
	\begin{definition}[prodotto scalare/hermitiano]
	dato $V$ spazio vettoriale sul campo $\mathbb{R}$ ($\mathbb{C}$) si definisce prodotto scalare (hermitiano) un'applicazione $\varphi: V \times V \to \mathbb{R}$ ($\varphi: V \times V \to \mathbb{C}$) che gode delle seguenti proprietà:
	\begin{enumerate}[label=\protect\circled{\arabic*}]
		\item lineare nella prima componente, ovvero
		$$
		\forall \underline{v}, \underline{w}, \underline{z}, \forall \alpha, \beta \in \mathbb{R}, \varphi(\alpha \underline{v} + \beta \underline{w}, \underline{z}) = \alpha \varphi(\underline{v}, \underline{z}) + \beta \varphi(\underline{w}, \underline{z})
		$$
		\item lineare nella seconda componente, ovvero
		$$
		\forall \underline{v}, \underline{w}, \underline{z}, \forall \alpha \beta \in \mathbb{R}, \varphi(\underline{v}, \alpha \underline{w} + \beta \underline{z}) = \alpha \varphi(\underline{v}, \underline{w}) + \beta \varphi(\underline{v}, \underline{z})
		$$
		\item simmetrica, ovvero
		$$
		\forall \underline{v}, \underline{w} \in V \, \, , \varphi(\underline{v}, \underline{w}) = \varphi(\underline{w}, \underline{v})
		$$
	\end{enumerate}
	(sostituendo $\mathbb{C}$ al posto di $\mathbb{R}$ si ottengono le proprietà che rendono identificano un prodotto hermitiano)
	\end{definition}
	\begin{definition}[spazio euclideo]
		uno spazio vettoriale $V$ munito di un prodotto scalare $\varphi: V \times V \to \mathbb{R}$ (analogamente nel caso di un prodotto hermitiano) definito positivo ($\forall \underline{v} \neq \underline{0}, \varphi(\underline{v}, \underline{v}) > 0$) si dice \textbf{spazio euclideo}
	\end{definition}
	\begin{remark}
	Quando un prodotto scalare/hermitiano è definito positivo, diciamo che è \textbf{coercivo}. Durante questo documento capiterà spesso di riferirci a questa proprietà con questo termine
	\end{remark}
	\noindent Introduciamo qualche notazione a noi utile:
	\begin{itemize}
		\item $E_n$: spazio affine euclideo di dimensione finita $n$;
		\item $\mathbb{E}_n$: spazio vettoriale euclideo, ottenuto fissando un'origine in $E_n$;
	\end{itemize}
	Sappiamo, dal corso di Geometria, che, fissando una base $\mathcal{B}$ di $\mathbb{E}_n$, le coordinate di un generico vettore $\underline{x} \in \mathbb{E}_n$ sono univoche e definendo la funzione $\varphi: \mathbb{E}_n \to \mathbb{R}^n$ tale che $\underline{x} \in \mathbb{E}_n \mapsto [\underline{x}]_{\mathcal{B}}$, dove con $[\underline{x}]_{\mathcal{B}}$ indichiamo il vettore delle coordinate del vettore $\underline{x}$ rispetto alla base $\mathcal{B}$ nello spazio $\mathbb{E}_n$. \\
Per comodità, noi vorremmo che questa base fosse anche ortonormale per semplificare la trattazione (l'esistenza è garantita, ovviamente, dal teorema di Lagrange). 
\begin{definition}[base ortonormale]
	sia $\mathcal{B} = \{ \underline{v}_1, \ldots \underline{v}_n \}$, diciamo che essa è una base ortonormale se
	$$
	\innerprod {\underline{v}_i} {\underline{v}_j} = \delta_{ij} = \begin{cases} 1 & i = j \\ 0 & i \neq j \end{cases}	
	$$
\end{definition}
\begin{remark}
	Se con la norma indotta dal prodotto scalare lo spazio è completo, allora $\mathbb{E}_n$ è uno spazio di Hilbert.
\end{remark}
\noindent Indichiamo adesso fissata la base ortonormale $(\underline{e}_1, \ldots \underline{e}_n)$. Osserviamo che, dati $\underline{v}, \underline{w}$, allora
$$
\innerprod {\underline{v}}{\underline{w}} = \innerprod{\sum_{j=1}^n x_j \underline{v}_j}{\sum_{i=1}^n y_i \underline{v}_i} = \sum_{j=1}^n \sum_{i=1}^n x_j y_i \innerprod{\underline{v}_j}{\underline{v}_i} = \sum_{j=1}^n \sum_{i=1}^n x_i y_i \delta_{ij} = \sum_{i=1}^n x_i y_i
$$
\begin{definition}[norma]
	Sia $V$ uno spazio vettoriale reale (o complesso). Si definisce norma un'applicazione $|| \cdot ||: V \to \mathbb{R}$ ($|| \cdot ||: V \to \mathbb{C}$) che verifica le seguenti condizioni:
	\begin{enumerate}[label=\protect\circled{\arabic*}]
		\item $||\underline{v}|| \geq 0 \, \, \forall \underline{v} \in V$
		\item $|| \underline{v} || = 0 \iff \underline{v} = \underline{0}$
		\item $|| \lambda \underline{v} || = | \lambda | || \underline{v} || \, \, \forall \lambda \in \mathbb{R}, \forall \underline{v} \in V$
		\item $|| \underline{v} + \underline{w} || \leq || \underline{v} || + || \underline{w} || \, \, \forall \underline{v}, \underline{w} \in V$
	\end{enumerate}
	(analogamente per uno spazio definito su $\mathbb{C}$ sostituendo $\mathbb{C}$ a $\mathbb{R}$)
\end{definition}
\noindent \textbf{Notazione}: la norma di un vettore verrà indicata all'interno di questo documento, per comodità, sia con $||\cdot||$ e sia con $|\cdot|$
\begin{remark}
Osserviamo che se $(V, \varphi)$ è uno spazio euclideo allora $|| \cdot || = \sqrt{\varphi(\cdot, \cdot)}$ è una norma: il prodotto scalare \emph{induce} una norma su $V$. Mostreremo più avanti il punto $\circled{4}$ (ovvero la proprietà \emph{meno} banale fra quelle) usando la norma indotta dal prodotto scalare
\end{remark}
\noindent Sapendo che $\mathbb{E}_n \simeq \mathbb{R}^n$(quando quest'ultimo è munito di un prodotto scalare definito positivo naturalmente) possiamo effettuare tutte le dimostrazioni in $\mathbb{R}^n$ e queste saranno naturalmente valide in tutti gli spazi euclidei di dimensione finita $n$. 
\begin{theorem} Sia $(V, \varphi)$ uno spazio euclideo. Allora $$\forall \underline{v}, \underline{w} \in V, \innerprod{\underline{v}}{\underline{w}} = ||\underline{v}|| \, || \underline{w} || \cos{\hat{\theta}}$$
dove $\hat{\theta}$ è l'angolo convesso fra i due vettori $\underline{v}$ e $\underline{w}$
\end{theorem}
\begin{proof}
consideriamo due vettori $\underline{v} \in \mathbb{E}_2$ con $|| \underline{v} || = || \underline{w} || = 1$. Prendiamo per semplicità $\underline{v} = \underline{e}_1$ e si osserva che
$$
\innerprod{\underline{e}_1}{\underline{w}} = \cos{\hat{\theta}}
$$
Questo segue banalmente dall'interpretazione geometrica del prodotto scalare canonico. \\
Per estendere la validità di questo risultato a tutti i $\underline{v} \neq \underline{e}_1$ si osserva che $\exists R \in \text{SO}(\mathbb{E}_2): R\underline{e}_1 = \underline{v}$ e dunque, considerando $R^{-1}(\underline{w})$ (l'esistenza di un'inversa è garantita dal fatto che $R \in \text{SO}(2)$) e sappiamo che:
$$
\innerprod{\underline{e}_1}{R^{-1}(\underline{w})} = \innerprod{R \underline{e}_1}{(R \circ R^{-1})\underline{w}} = \cos{\hat{\theta}}
$$
e, siccome 
A questo punto, dati due vettori qualunque $\underline{v}$ e $\underline{w}$ non di norma unitaria, possiamo utilizzare il ragionamento procedente osservando che:
$$
\innerprod{\frac{\underline{v}}{|\underline{v}|}} {\frac{\underline{w}}{|\underline{w}|}} = \cos{\hat{\theta}}
$$
ma allora, si osserva che:
$$
\frac{1}{|\underline{v}|} \cdot \frac{1}{|\underline{w}|} \innerprod{\underline{v}} {\underline{w}} = \cos{\hat{\theta}} \implies \innerprod{\underline{v}}{ \underline{w}} = |\underline{v}| |\underline{w}| \cos{\hat{\theta}}
$$
dunque la tesi (in $\mathbb{E}_2$). Per generalizzare questo concetto a qualunque spazio, noi sappiamo che possiamo considerare il piano $\pi = \text{Span}(\underline{v}, \underline{w})$ e considerare il loro angolo $\hat{\theta}$ convesso giacente in questo piano.
\end{proof}
\begin{prop}[Disuguaglianza di Cauchy-Schwarz]
$$\forall \underline{v}, \underline{w} \in \mathbb{E}_n, \, | \innerprod{\underline{v}}{\underline{w}} | \leq ||\underline{v}|| \, ||\underline{w}||$$
\end{prop}
\begin{proof}
Consideriamo $\lambda \in \mathbb{R}$ e sappiamo che, per coercività del prodotto scalare, che:
$$
\forall \underline{v}, \underline{w} \in \mathbb{E}_n, \, \innerprod {\underline{v} + \lambda \underline{w}} {\underline{v} + \lambda \underline{w}} > 0
$$
ma per bilinearità del prodotto scalare abbiamo che
$$
\innerprod {\underline{v} + \lambda \underline{w}} {\underline{v} + \lambda \underline{w}} = \innerprod{\underline{v}}{\underline{v}} + 2\lambda \innerprod{\underline{v}} {\underline{w}} + \lambda^2 \innerprod{\underline{w}}{\underline{w}}> 0
$$
dunque l'equazione in $\lambda$
$$
\innerprod{\underline{v}}{\underline{v}} + 2\lambda \innerprod{\underline{v}} {\underline{w}} + \lambda^2 \innerprod{\underline{w}}{\underline{w}} = 0
$$
non deve avere soluzione, il che implica che
$$
\Delta = 4\innerprod{\underline{v}}{\underline{w}}^2 - 4 \innerprod{\underline{v}}{\underline{v}} \innerprod{\underline{w}}{\underline{w}} < 0 \implies \innerprod{\underline{v}}{\underline{w}}^2 < \innerprod{\underline{v}}{\underline{v}} \innerprod{\underline{w}}{\underline{w}} \implies |\innerprod{\underline{v}}{\underline{w}}| < ||\underline{v}|| \, ||\underline{w}||
$$
\end{proof}
\begin{prop}[Disuguaglianza triangolare] \hfill \\
$ \forall \underline{v}, \underline{w} \in \mathbb{E}_n, $
\begin{enumerate}[label=\protect\circled{\arabic*}]
	\item $|\underline{v} + \underline{w}| \leq |\underline{v}| + |\underline{w}|$
	\item $||\underline{v}| - |\underline{w}|| \leq |\underline{v} - \underline{w}|$
\end{enumerate}
\label{prop:dis_triang}
\end{prop}
\begin{proof}
osserviamo che
$$
|\underline{v} + \underline{w}|^2 = \innerprod{\underline{v} + \underline{w}}{\underline{v} + \underline{w}} = \innerprod{\underline{v}}{\underline{v}} + 2\innerprod{\underline{v}}{\underline{w}} + \innerprod{\underline{w}}{\underline{w}} = |\underline{v}|^2 + 2 \innerprod{\underline{v}}{\underline{w}} + |\underline{w}|^2 \stackrel{\text{dis. di Cauchy-Schwarz}}{\leq} |\underline{v}|^2 + 2|\underline{v}| \, |\underline{w}| + |\underline{w}|^2
$$
dunque
$$
|\underline{v} + \underline{w}|^2 \leq |\underline{v}|^2 + 2|\underline{v}| \, |\underline{w}| + |\underline{w}|^2 = (|\underline{v}| + |\underline{w}|)^2 \implies |\underline{v} + \underline{w}| \leq |\underline{v}| + |\underline{w}|
$$
e si ottiene la tesi del punto $\circled{1}$. \\
Il punto $\circled{2}$ si ottiene come corollario del primo punto osservando che
$$
|\underline{v}| = |\underline{v} + \underline{w} - \underline{w}| \leq |\underline{v} - \underline{w}| + |\underline{w}| \implies |\underline{v}| - |\underline{w}| \leq |\underline{v} - \underline{w}| 
$$
ma ragionando in maniera identica sul vettore $\underline{w}$ si osserva che
$$
|\underline{w}| = |\underline{w} - \underline{v} + \underline{v}| \leq |\underline{w} - \underline{v}| + |\underline{v}| \implies |\underline{w}| - |\underline{v}| \leq |\underline{w} - \underline{v}|= |\underline{v} - \underline{w}|
$$
dunque possiamo concludere che
$$
||\underline{v}| - |\underline{w}|| \leq |\underline{v} - \underline{w}|
$$
ottenendo la tesi
\end{proof}
\begin{remark}
Come avevo detto in una osservazione, ogni prodotto scalare definito positivo induce sempre una norma. La dimostrazione qua sopra non fa uso di nessuna proprietà specifiche del prodotto scalare canonico di $\mathbb{R}^n$, dunque può essere usata per ogni norma indotta dal prodotto scalare definito positivo di qualunque spazio euclideo
\end{remark}
\section{Il prodotto vettoriale}
\noindent In $\mathbb{R}^3$ (ma in generale in qualunque spazio euclideo di dimensione $3$) è anche possibile definire l'operazione di prodotto vettoriale, molto utile per trattare (come vedremo più avanti) l'orientazione delle superfici. \\
Fissando una base ortonormale $\{\underline{e_1}, \underline{e_2}, \underline{e_3} \}$ di $\mathbb{E}_3$, definendo questa operazione $\times: V \times V \to V$ assegnando i prodotti elementari secondo l'invarianza per permutazioni cicliche, ponendo che
$$
\underline{e_1} \times \underline{e_2} = \underline{e_3}
$$
e, per invarianza per permutazioni cicliche, dovremo avere che
\begin{align*}
&\underline{e_3} \times  \underline{e_1} = \underline{e_2} \\
&\underline{e_2} \times \underline{e_3} = \underline{e_1}
\end{align*}
Le permutazioni non cicliche invece fanno variare il segno dunque avremo, per il vettore $\underline{e_1}$
\begin{align}
	\begin{cases}
		\underline{e_1} \times \underline{e_2} = \underline{e_3} \\
		\underline{e_1} \times \underline{e_3} = - \underline{e_2} \\
		\underline{e_2} \times \underline{e_3} = \underline{e_1}
	\end{cases}
\end{align}
Le regole che abbiamo visto possono essere anche facilmente ottenute tramite la cosiddetta \emph{regola della mano destra}: indicando la direzione del primo vettore con il pollice (in questo caso $\underline{e_1}$) e con l'indice il secondo vettore (in questo caso $\underline{e_2}$), ottenendo sul pollice la direzione del terzo vettore. \\
Vogliamo inoltre che questo prodotto vettoriale sia \emph{bilineare}. \\
Dati adesso due vettori $\underline{v}$ e $\underline{w} \in E_3$ abbiamo che $\underline{v}, \underline{w} \in \text{Span}(\underline{e_1}, \underline{e_2}, \underline{e_3})$ dunque avremo che:
\begin{align*}
&\underline{v} = x_1 \underline{e_1} + x_2 \underline{e_2} + x_3 \underline{e_3} &
&\underline{w} = y_1 \underline{e_1} + y_2 \underline{e_2} + y_3 \underline{e_3}
\end{align*}
e studiamo quali saranno le coordinate del prodotto vettoriale fra questi due:
\begin{align*}
&\underline{v} \times \underline{w} =  (x_1 \underline{e_1} + x_2 \underline{e_2} + x_3 \underline{e_3}) \times (y_1 \underline{e_1} + y_2 \underline{e_2} + y_3 \underline{e_3}) = \\
	&= (x_1 y_2 - x_2 y_1)( \underline{e_1} \times \underline{e_2}) + (x_1 y_3 - x_3 y_1) (\underline{e_1} \times \underline{e_3}) + (x_2 y_3 - x_3 y_2) (\underline{e_2} \times \underline{e_3})
\end{align*}
dunque definiamo una matrice $C$ di questa forma:
\begin{align}
C = \begin{pmatrix}
	x_1 & y_1 \\
	x_2 & y_2 \\
	x_3 & y_3 
\end{pmatrix}
\end{align}
e ne consideriamo i minori:
\begin{align}
M_{ij}(C) = \text{det}\begin{pmatrix}
	x_i & y_i \\
	x_j & y_j
\end{pmatrix} = x_i y_j - x_j y_i
\end{align}
dunque otteniamo una nuova formula:
$$
\underline{v} \times \underline{w} = M_{12}(C)(\underline{e_1} \times \underline{e_2}) + M_{23}(C) (\underline{e_2} \times \underline{e_3}) + M_{13}(C) (\underline{e_1} \times \underline{e_3}) = M_{12}(C)\underline{e_3} + M_{23}(C)\underline{e_1} - M_{13}(C) \underline{e_2}
$$
usando le proprietà del determinante, sappiamo che $M_{13}(C) = -M_{31}(C)$ dunque
$$
\underline{v} \times \underline{w} = M_{12}(C) \underline{e_3} + M_{23}(C) \underline{e_1} + M_{31} \underline{e_2}
$$
abbiamo dunque dimostrato la seguente proposizione
\begin{prop} dati i vettori $\underline{v}, \underline{w} \in \mathbb{E}_3$ allora $\exists x_1, \ldots, x_3, y_1, \ldots y_3 \in \mathbb{R}$ tali che $\underline{v} = x_1 \underline{e_1} + x_2 \underline{e_2} + x_2 \underline{e_3}$ e $\underline{w} = y_1 \underline{e_1} + y_2 \underline{e_2} + y_3 \underline{e_3}$. Posta la matrice $C$ tale che
$$
C = \begin{pmatrix}
	x_1 & y_1 \\
	x_2 & y_2 \\
	x_3 & y_3
\end{pmatrix}
$$
allora
	\begin{equation}
\underline{v} \times \underline{w} = M_{23}(C)\underline{e_1} + M_{31}(C)\underline{e_2} + M_{12}(C)\underline{e_3}
\end{equation}
\end{prop}
\begin{remark}
si noti come i numeri siano tutti disposti secondo permutazioni cicliche: questo può essere un buon trucco per ricordarsela.
\end{remark}
\begin{cor}[matrice del prodotto vettoriale] siano $\underline{v}, \underline{w} \in \mathbb{E}_3$ e siano $x_1, \ldots x_3$ e $y_1, \ldots y_3$ le rispettive coordinate rispetto alla base ortonormale $\mathcal{B} = \{ \underline{e_1}, \underline{e_2}, \underline{e_3}\}$ di $\mathbb{E}_3$. Allora
\begin{equation}
\underline{v} \times \underline{w} = \text{det}\begin{pmatrix}
	x_1 & y_1 & \underline{e_1} \\
	x_2 & y_2 & \underline{e_2} \\
	x_3 & y_3 & \underline{e_3}
\end{pmatrix}
\end{equation}
\label{cor:pr_vett_det}
\end{cor}
\begin{proof}
si osservi che, procedendo con uno sviluppo di Laplace lungo la terza colonna abbiamo che:
$$
\text{det}\begin{pmatrix}
x_1 & y_1 & \underline{e_1} \\
	x_2 & y_2 & \underline{e_2} \\
	x_3 & y_3 & \underline{e_3}
\end{pmatrix} = \underline{e_1} M_{23}(C) - \underline{e_2}M_{13}(C) + \underline{e_3} M_{12}(C) = M_{23}(C) \underline{e_1} + M_{31}(C) \underline{e_2} + M_{12}(C) \underline{e_3}
$$
\end{proof}
\begin{prop}[matrice del prodotto misto]
siano dati i vettori $\underline{v}=x_1 \underline{e_1} + x_2 \underline{e_2} + x_3 \underline{e_3}$ e $\underline{w} = y_1 \underline{e_1} + y_2 \underline{e_2} + y_3 \underline{e_3}$ e $\underline{z} = \xi_1 \underline{e_1} + \xi_2 \underline{e_2} + \xi_3 \underline{z_3}$. Allora
\begin{equation}
\innerprod{\underline{v} \times \underline{w}}{\underline{z}} = \text{det}\begin{pmatrix}
	x_1 & y_1 & \xi_1 \\
	x_2 & y_2 & \xi_2 \\
	x_3 & y_3 & \xi_3
\end{pmatrix}
\end{equation}
\label{prop:prod_mist}
\end{prop}
\begin{proof}
	sappiamo che $\underline{v} \times \underline{w} = M_{23}(C)\underline{e_1} + M_{31}(C) \underline{e_2} + M_{12}(C) \underline{e_3}$ dunque $\innerprod{\underline{v} \times \underline{w}}{\underline{z}} = M_{23}(C)\xi_1 + M_{31}(C)\xi_2 + M_{12}(C)\xi_3$. D'altra parte definendo la matrice 
	$$M=\begin{pmatrix}
		x_1 & y_1 & \xi_1 \\
		x_2 & y_2 & \xi_2 \\
		x_3 & y_3 & \xi_3
	\end{pmatrix} 
$$
allora si osserva che
$$
\innerprod{\underline{v} \times \underline{w}}{\underline{z}} = \text{cof}_{13}(M)\xi_1 + \text{cof}_{23}(M)\xi_2 + \text{cof}_{33}(M) \xi_3 = \text{det}(M)
$$
dunque la tesi.
\end{proof}
\begin{cor}[$\underline{v} \times \underline{w} \perp \text{Span}(\underline{v}, \underline{w})$]
Dati due vettori $\underline{v}, \underline{w} \in \mathbb{E}_3$ allora
$$
\underline{v} \times \underline{w} \, \perp \text{Span}(\underline{v}, \underline{w})
$$
\end{cor}
\begin{proof}
dalla proposizione \ref{prop:prod_mist} sappiamo che
$$
\innerprod{\underline{v} \times \underline{w}}{\underline{v}} = \text{det}\begin{pmatrix}
	x_1 & y_1 & x_1 \\
	x_2 & y_2 & x_2 \\
	x_3 & y_3 & x_3
\end{pmatrix} = 0
$$
per proprietà del determinante. Similmente per il vettore $\underline{w}$. Dunque abbiamo che $\underline{v} \times \underline{w} \perp \underline{v}$ e $\underline{v} \times \underline{w} \perp \underline{w}$ che implica che $\underline{v} \times \underline{w}$ è perpendicolare a qualunque combinazione lineare dei vettori $\underline{v}$ e $\underline{w}$, quindi $\underline{v} \times \underline{w} \perp \text{Span}(\underline{v}, \underline{w})$ ovvero la tesi.
\end{proof}
\begin{cor}[norma del prodotto vettoriale]
Siano $\underline{v}, \underline{w} \in \mathbb{E}_3$ con $\underline{v} = x_1 \underline{e_1} + x_2 \underline{e_2} + x_3 \underline{e_3}$ e $\underline{w} = y_1 \underline{e_1} + y_2 \underline{e_2} + y_3 \underline{e_3}$. Allora, ponendo
$$
C = \begin{pmatrix}
	x_1 & y_1 \\
	x_2 & y_2 \\
	x_3 & y_3 
\end{pmatrix}
$$
allora
$$
|\underline{v} \times \underline{w}| = \sqrt{M_{23}(C)^2 + M_{12}(C)^2 + M_{31}(C)^2} 
$$
\label{cor:norm_cross_prod}
\end{cor}
\begin{proof}
Osserviamo, per la proposizione \ref{prop:prod_mist}, che
\begin{align*}
&\innerprod{\underline{v} \times \underline{w}}{\underline{v} \times \underline{w}} = \text{det}\begin{pmatrix}
\underline{v} & \underline{w} & \underline{v} \times \underline{w}
\end{pmatrix} = M_{23}(C) (\underline{v} \times \underline{w})_1 + M_{31}(C) (\underline{v} \times \underline{w})_2 + M_{12}(C) (\underline{v} \times \underline{w})_3 = \\
&= M_{23}(C)^2 + M_{31}(C)^2 + M_{12}(C)^2
\end{align*}
\noindent dunque $|\underline{v} \times \underline{w}| = \sqrt{M_{23}(C)^2 + M_{31}(C)^2 + M_{12}(C)^2}$
\end{proof}
\begin{prop}
Sia $R \in \text{SO}(3)$ e siano dati $\underline{v}, \underline{w} \in \mathbb{E}_3$. Allora
\begin{equation}
R(\underline{v} \times \underline{w}) = R\underline{v} \times R\underline{w}
\end{equation}
\end{prop}
\begin{proof}
Sia $\underline{u} \in \mathbb{E}_3$ e consideriamo il seguente prodotto scalare:
\begin{align*}
\innerprod{R\underline{v} \times R\underline{w}}{R\underline{u}} &= \text{det}\begin{pmatrix}
R\underline{v} & R\underline{w} & R\underline{u}
\end{pmatrix} \\
&= \text{det}\left(R \begin{pmatrix}
\underline{v} & \underline{w} & \underline{u} 
\end{pmatrix}\right)
\end{align*}
Usando il teorema di Binet abbiamo che
$$
\innerprod{R\underline{v} \times R\underline{w}}{R\underline{u}} = \text{det}\left(R \begin{pmatrix}
\underline{v} & \underline{w} & \underline{u} 
\end{pmatrix}\right) = \text{det}(R)\text{det}\begin{pmatrix}
\underline{v} & \underline{w} & \underline{u} 
\end{pmatrix} = \text{det}\begin{pmatrix}
	\underline{v} & \underline{w} & \underline{u}
\end{pmatrix}
$$
dove abbiamo usato l'ipotesi che $R \in \text{SO}(3)$ dunque $\text{det}(R) = 1$ e siccome sappiamo che $R$ è un'isometria allora
$$
\innerprod{R(\underline{v} \times \underline{w})}{R\underline{u}} = \text{det}\begin{pmatrix} \underline{v} & \underline{w} & \underline{u} \end{pmatrix} = \innerprod{\underline{v} \times \underline{w}}{\underline{u}} = \innerprod{R(\underline{v} \times \underline{w})}{R\underline{u}}
$$
ma allora $\innerprod{R\underline{v} \times R\underline{w}}{R\underline{u}} = \innerprod{R(\underline{v} \times \underline{w})}{R\underline{u}} \implies \innerprod{R(\underline{v} \times \underline{w}) - (R\underline{v} \times R\underline{w})}{R\underline{u}} = 0$. Ma allora, per la coercività del prodotto scalare, abbiamo necessariamente che
$$
R(\underline{v} \times \underline{w}) = R\underline{v} \times R\underline{w}
$$
\end{proof}
\begin{prop}
Se $\underline{v}, \underline{w} \in \mathbb{R}^3, \underline{v} \perp \underline{w}$ e $|\underline{v}| = |\underline{w}| = 1$ allora
$$
|\underline{v} \times \underline{w}| = 1
$$
\label{prop:norm_unit_cross_prod}
\end{prop}
\begin{proof}
Dati $\underline{v}$ e $\underline{w}$ con $\underline{v} \perp \underline{w}$, allora $\{\underline{v}, \underline{w}, \underline{v} \times \underline{w} \}$ è una base. Siccome appartengono a $\mathbb{R}^3$ e sono perpendicolari, allora sappiamo che $\exists R \in \text{SO}(3): R\underline{v} = \underline{e_1}$, $R\underline{w} = \underline{e_2}$ e $R(\underline{v} \times \underline{w}) = \underline{e_3}$ dunque
$$
|\underline{v} \times \underline{w}| = 1 = |R\underline{e_1} \times R\underline{e_2}|= |R(\underline{e_1} \times \underline{e_2})| = |\underline{v} \times \underline{w}| \implies |\underline{v} \times \underline{w}| = 1
$$
\end{proof}
\begin{cor}[modulo del prodotto vettoriale di vettori perpendicolari]
siano $\underline{v}, \underline{w} \in \mathbb{E}_3$ e $\underline{v} \perp \underline{w}$ allora:
$$
	|\underline{v} \times \underline{w}| = |\underline{v}| \, |\underline{w}|
$$
\end{cor}
\begin{proof}
siccome $\underline{v} \perp \underline{w}$ allora $\frac{\underline{v}}{|\underline{v}|} \perp \frac{\underline{w}}{|\underline{w}|}$. Ma allora possiamo applicare a questi due vettori la proposizione precedente, dunque:
$$
\Big|\frac{\underline{v}}{|\underline{v}|} \times \frac{\underline{w}}{|\underline{w}|} \Big| = 1 = \frac{1}{|\underline{v}| \, |\underline{w}|} |\underline{v} \times \underline{w}| \implies |\underline{v} \times \underline{w}| = |\underline{v}| \, |\underline{w}|
$$
\end{proof}
\begin{prop}[area del parallelogramma]
Siano $\underline{v}, \underline{w} \in \mathbb{E}_3$. Allora $$|\underline{v} \times \underline{w}| = |\underline{v}| \, |\underline{w}| \sin{\hat{\theta}}$$
dove $\hat{\theta}$ è l'angolo convesso fra i due vettori
\end{prop}
\begin{proof}
possiamo "ortogonalizzare" il vettore $\underline{v}$ usando il procedimento di Grand-Schmit, dunque
\begin{align*}|\underline{v} \times \underline{w}|^2 = \Bigg| &\left( \underline{v} - \frac{\innerprod{\underline{v}}{\underline{w}}}{|\underline{w}|^2} \underline{w} \right) \times \underline{w} \Bigg|^2 = \Bigg| \underline{v} - \frac{\innerprod{\underline{v}}{\underline{w}}}{|\underline{w}|^2} \underline{w} \Bigg|^2 |\underline{w}|^2 = \innerprod{\underline{v} - \frac{\innerprod{\underline{v}}{\underline{w}}}{|\underline{w}|^2} \underline{w}}{\underline{v} - \frac{\innerprod{\underline{v}}{\underline{w}}}{|\underline{w}|^2} \underline{w}} |\underline{w}|^2 = \\
&= |\underline{w}|^2 \left( \innerprod{\underline{v}}{\underline{v}} - 2 \innerprod{\underline{v}}{\frac{\innerprod{\underline{v}}{\underline{w}}}{|\underline{w}|^2} \underline{w}} + \frac{(\innerprod{\underline{v}}{\underline{w}})^2}{|\underline{w}|^2} \innerprod{\underline{w}}{\underline{w}} \right) = \\
&= |\underline{w}|^2 \left( \innerprod{\underline{v}}{\underline{v}} - 2 \frac{\innerprod{\underline{v}}{\underline{w}}}{|\underline{w}|^2} \innerprod{\underline{v}}{\underline{w}} + \frac{(\innerprod{\underline{v}}{\underline{w}})^2}{|\underline{w}|^2} \innerprod{\underline{w}}{\underline{w}} \right) = \\
&= |\underline{w}|^2 \left( \innerprod{\underline{v}}{\underline{v}} - \frac{(\innerprod{\underline{v}}{\underline{w}})^2}{|\underline{w}|^2} \right)
\end{align*}
Sappiamo adesso che $\innerprod{\underline{v}}{\underline{w}} = |\underline{v}| \, |\underline{w}|\cos{\hat{\theta}}$, dunque
$$
|\underline{w}|^2 \left(|\underline{v}|^2 - |\underline{v}^2|\cos^2{\hat{\theta}} \right) = |\underline{v}|^2 |\underline{w}|^2 (1-\cos^2{\hat{\theta}}) = |\underline{v}|^2 |\underline{w}|^2 \sin^2{\hat{\theta}}
$$
ma allora se ne conclude che
$$
|\underline{v} \times \underline{w}|^2 = |\underline{v}|^2 |\underline{w}|^2 \sin^2{\hat{\theta}} \implies |\underline{v} \times \underline{w}| = |\underline{v}| \, |\underline{w}|\sin{\hat{\theta}}
$$
e questo conclude la dimostrazione.
\end{proof}
\begin{remark}
Unendo quest'ultima proposizione con il corollario \ref{cor:norm_cross_prod} possiamo facilmente vedere che
$$
|\underline{v} \times \underline{w}| = |\underline{v}| \, |\underline{w}| \sin{\hat{\theta}} = \sqrt{M_{23}(C)^2 + M_{12}(C)^2 + M_{31}(C)^2}
$$
\end{remark}
\begin{definition}[base positivamente orientata]
diremo che tre vettori $\underline{v}, \underline{w}, \underline{z} \in \mathbb{E}_3$ linearmente indipendenti sono una base positivamente orientata se
$$
\text{det}\begin{pmatrix}
	\underline{v} & \underline{w} & \underline{z}
\end{pmatrix} > 0
$$
\end{definition}
\begin{remark}
La base canonica di $\mathbb{R}^3$ è positivamente orientata, siccome
$$
\text{det}\begin{pmatrix}
	1 & 0 & 0 \\
	0 & 1 & 0 \\
	0 & 0 & 1
\end{pmatrix} = \text{det}(I) = 1
$$
\end{remark}
\begin{prop}
	Se $\underline{v}, \underline{w} \in \mathbb{R}^3$ sono linearmente indipendenti, allora l'insieme $\mathcal{B} = \{\underline{v}, \underline{w}, \underline{v} \times \underline{w} \}$ è una base positivamente orientata
\label{prop:base_pos_orient}
\end{prop}
\begin{proof}
banalmente
$$
\text{det}\begin{pmatrix}
	\underline{v} & \underline{w} & \underline{v} \times \underline{w} \end{pmatrix} = \innerprod{\underline{v} \times \underline{w}}{\underline{v} \times \underline{w}} = |\underline{v} \times \underline{w}|^2 >  0
$$
\end{proof}
\begin{remark}
Una base positivamente orientata si può sempre rappresentare con la regola della mano destra: il vettore $\underline{v}$ è rappresentato dall'indice, il vettore $\underline{w}$ è rappresentato dal dito medio e il pollice invece rappresenta $\underline{v} \times \underline{w}$.
\end{remark}
\noindent Infine, per concludere per adesso la lista dei teoremi sul prodotto vettoriale, abbiamo che:
\begin{prop}
Una base ortonormale $\{ \underline{v} , \underline{w} , \underline{v} \times \underline{w} \}$ è positivamente orientata se e solo se $\underline{z} = \underline{v} \times \underline{w}$
\end{prop}
\begin{proof} \hfill \\
$\boxed{\Leftarrow} \,$: già mostrato per la proposizione \ref{prop:base_pos_orient} \\
$\boxed{\Rightarrow} \,$: supponiamo di avere la seguente base ortonormale positivamente orientata $\{\underline{i}, \underline{j}, \underline{k} \}$, allora abbiamo che questa segue la regola della mano destra, dunque $\underline{i} \times \underline{j} = \alpha \underline{k}$ con $k \in \mathbb{R}$. Ma allora, per la proposizione \ref{prop:norm_unit_cross_prod}, abbiamo che $|\alpha \underline{k}|=1$ (siccome $\underline{v} \perp \underline{w}$) dunque $\underline{i} \times \underline{j} = \underline{k}$. 	
\end{proof}
\section{Topologia di $\mathbb{R}^n$}
Come visto dal corso di Analisi Matematica del primo anno, possiamo pensare $\mathbb{R}^n$ come uno spazio metrico se consideriamo la distanza indotta dal prodotto scalare. \\
Possiamo dunque andare a definire le \textbf{palle}:
\begin{definition}[palle aperte e chiuse]
Sia $x \in \mathbb{R}^n$ e fissiamo $r > 0$. Allora definiamo la palla aperta di raggio $r$ nella seguente maniera
$$
B(x, r) = \{ z \in \mathbb{R}^n: |z-x| < r \}
$$
e la palla chiusa di raggio $r$ nella seguente maniera
$$
\mathbb{B}(x, r) = \{ z \in \mathbb{R}^n: |z - x| \leq r \}
$$
con centro $x$ e raggio $r$.
\end{definition}
\noindent Ma cosa vuol dire palla aperta? In generale possiamo definire la nozione di insieme aperto nella seguente maniera:
\begin{definition}[insieme aperto]
$A \subset \mathbb{R}^n$ è un insieme aperto se $\forall y \in A \exists r > 0: B(\underline{y}, r_y) \subset A$
\end{definition}
\begin{exercise}
Mostrare che $B(x, r)$ è un aperto $\forall x \, \forall r$
\end{exercise}
\begin{proof}[Svolgimento]
Sia $x \in \mathbb{R}^n$ e sia fissato $r > 0$. Allora scelto $y \in B(x, r)$ sappiamo che preso $\varepsilon = r - |x - y| > 0$ allora 
possiamo definire la palla $B(y, \varepsilon)$ e prendere $z \in B(y, \varepsilon)$ e osservare che
$$
|z-x| \leq |z-y| + |y-x| < r - |x-y| + |x-y| = r
$$
dunque $|z-x| < r$ il che implica che $B(y, \varepsilon) \subset B(x, r)$
\end{proof}
\begin{exercise}
Siano $\{A_j: j \in J \}$ una famiglia di aperti (non necessariamente numerabile), allora $\bigcup\limits_{j \in J} A_j$ è un insieme aperto.
\end{exercise}
\begin{proof}[Svolgimento]
si osserva che, siccome $A_j$ è un insieme aperto $\forall j$, allora $\forall a \in A_j \exists r > 0: B(a, r) \subset A_j$, ma allora $B(a, r) \subset \bigcup\limits_{j \in J} A_j$, dunque $\forall a \in \bigcup\limits_{j \in J} A_j \exists r > 0: B(a, r) \subset \bigcup\limits_{j \in J} A_j$
\end{proof}

\chapter{Successioni e funzioni continue}
In questo capitolo andremo a definire il concetto di successione e vedere il forte collegamento presente fra esse e il concetto di continuità in più variabili. \\
Ricordiamo al lettore che una successione a valori in un insieme $A$ è una funzione $x_k: \mathbb{N} \to A$. Vediamo adesso come si definisce il concetto di \emph{successione convergente} in più variabili
\begin{definition}[convergenza di una successione]
diremo che $\{x_k \} \subset \mathbb{R}^n$ è una successione convergente a $z \in \mathbb{R}^n$ se $\lim\limits_{k \to +\infty} |x_k - z| = 0$.
\end{definition}
\begin{remark}
Naturalmente se $\underline{x_k} = (x_{k_1}, x_{k_2}, \ldots x_{k_n}) \in \mathbb{R}^n$ converge a $\underline{z} = (z_1, z_2, \ldots z_n) \iff \lim\limits_{k \to \infty} x_{k_i} = z_i \, \forall i \in 1, \ldots n$
\end{remark}
\begin{example}
Consideriamo la successione $\underline{x_k} = (e^{-k} + 1,(-1)^k)$. Per il precedente teorema questa successione è convergente.
\end{example}
\begin{prop}[Unicità del limite]
Il limite di una successione è unico. 
Se $\underline{x_k} \in \mathbb{R}^n \, \, \forall k$ converge a $\underline{z} \in \mathbb{R}^n$ allora $z$ è unico.
\end{prop}
\begin{proof}
Supponiamo per assurdo che $\underline{x_k}$ converga a $\underline{z}$ e $\underline{y}$ con $\underline{z} \neq \underline{y}$. Allora
$$
|\underline{z} - \underline{y}| \leq |\underline{x_k} - \underline{z}| + |\underline{x_k} - \underline{y}| \stackrel{k \to +\infty}{\to} 0 \implies z = y
$$
\end{proof}

\begin{prop}
Se $\underline{x_k} \to \underline{x}$ allora $|\underline{x_k}| \to |\underline{x}|$
\end{prop}
\begin{proof}
si osserva che, dalla proposizione \ref{prop:dis_triang}, si ha che
$$
||\underline{x_k}| - |\underline{x}|| \leq |\underline{x_k} - \underline{x}| \stackrel{k \to +\infty}{\to} 0
$$
\end{proof}
Mostriamo adesso una banale proposizione, le cui conseguenze sono tuttavia banali
\begin{prop}[spazio vettoriale delle successioni convergenti]
Se $\underline{x_k} \to \underline{x}$ e $\underline{y_k} \to \underline{y}$ allora $\forall \lambda, \mu \in \mathbb{R}, \, \lambda \underline{x_k} + \mu \underline{y_k} \to \lambda \underline{x} + \mu \underline{y}$ 
\end{prop}
\begin{proof}
osserviamo che
$$
|\lambda \underline{x_k} + \mu \underline{y_k} - \lambda \underline{x} - \mu \underline{y}| = |\lambda (\underline{x_k} - \underline{x}) + \mu (\underline{y_k} - \underline{y})|
$$
Per la proposizione \ref{prop:dis_triang} abbiamo che
$$
|\lambda (\underline{x_k} - \underline{x}) + \mu (\underline{y_k} - \underline{y})| \leq |\lambda| |\underline{x_k} - \underline{x}| + |\mu| |\underline{y_k} - \underline{y}| \stackrel{k \to +\infty}{\to} 0
$$
dunque la tesi
\end{proof}
\begin{remark}
l'importanza di questa dimostrazione sta nel fatto che questo teorema dimostra che l'insieme delle successioni convergenti in $\mathbb{R}^n$ forma uno spazio vettoriale che è chiuso rispetto all'addizione $+_{\mathbb{R}}$ e prendendo come prodotto per scalare $*_\mathbb{R}$
\end{remark}
\noindent Adesso andiamo a mostrare una proprietà che segue direttamente dalla topologia di $\mathbb{R}^n$ (in spazi topologici qualunque non è sempre valido)
\begin{prop}[caratterizzazione degli insiemi chiusi]
	Un insieme $A \subset \mathbb{R}^n$ è chiuso se e solo se $\forall \underline{x_k} \to \underline{z}, \underline{x_k} \in A \, \forall k$ allora $z \in A$. Formalmente
	$$
	(A \subset \mathbb{R}^n \text{è chiuso}) \iff (\forall \underline{x_k} \to \underline{z}, \underline{x_k} \in A \, \forall k \implies z \in A) 
	$$
\end{prop}
\begin{proof} \hspace{1em} \newline
$\boxed{\Rightarrow}$: Se $A$ è chiuso, considerando una generica $\underline{x_k} \to \underline{z}, \underline{x_k} \in A \, \forall k$, allora $\forall r > 0$ dato che $|\underline{x_k} - \underline{z}| \to 0 \, \exists k_r \in \mathbb{N}$ tale che $|\underline{x_k} - \underline{z}| < r \, \forall k \geq k_r \implies \underline{x_k} \in B(z, r) \implies B(z, r) \cap A \neq \emptyset \implies z \in \bar{A}$. Siccome A è chiuso allora $\bar{A} = A$ dunque $z \in A$. \\
$\boxed{\Leftarrow}$: Sia $\underline{w} \in \bar{A} \implies B(\underline{w}, \frac{1}{k}) \cap A \neq \emptyset \, \forall k \geq 1$ quindi esiste una successione $\underline{x_k}$ (potremmo prendere per esempio $\underline{w-\frac{1}{k}}$). Ma allora $|\underline{x_k} - \underline{w}| < \frac{1}{k} \implies \underline{x_k} = \underline{w} \implies \underline{w} \in A$ per la seconda proprietà, ma allora $\bar{A} \subset A \implies \bar{A} = A$ e dunque $A$ è chiuso per definizione. 
\end{proof}
\noindent Questa proprietà è molto importante, siccome garantisce l'equivalenza fra \textbf{chiusura sequenziale} e \textbf{chiusura} (che negli spazi topologici non è generalmente garantito) in $\mathbb{R}^n$. \\
In $\mathbb{R}^n$, dato un insieme $A \subset \mathbb{R}^n$, se vogliamo verificare che sia chiuso sarà dunque necessario verificare che ogni successione $\underline{x_k}$ tali che $\forall k, \underline{x_k} \in A$ avremo che $\underline{x_k} \to \underline{x} \in A$, ovvero che ogni successione convergente in $A$ converga ad un punto che appartiene ad $A$.
\section{Funzioni continue}
\begin{definition}[funzioni continue]
Una funzione $f: A \to \mathbb{R}$ con $A \subset \mathbb{R}^m$ si dice continua in $\underline{z} \in A$ se $$\forall \underline{x_k} \in A, \underline{x_k} \to \underline{z}, \, \lim\limits_{k \to +\infty} f(\underline{x_k}) = f(\underline{z})$$
\end{definition}
\begin{remark}
Diremo che $f: A \to \mathbb{R}$ con $A \subset \mathbb{R}^n$ è continua in $A$ se è continua $\forall \underline{x} \in A$.
\end{remark}
\noindent Andiamo a ricordare un concetto molto importante:
\begin{definition}[apertura relativa]
Fissato $A \subset \mathbb{R}^n$ diremo che $S \subset A$ è aperto in $A$ se $\exists \Omega \subset \mathbb{R}^n$ aperto tale che $$S = A \cap \Omega$$
\end{definition}
Dopo questa definizione siamo pronti per enunciare una serie di teoremi sulle funzioni continue che le caratterizzano in $\mathbb{R}^n$.
\begin{theorem}[teorema C1]
Sia $f: A \to \mathbb{R}^m$ con $A \subset \mathbb{R}^n$. Allora $f$ è continua $\iff \forall U \subset \mathbb{R}^m$ aperto $f^{-1}(U)$ è aperto in $A$
\end{theorem}
\begin{proof} \hspace{1em} \newline
$\boxed{\Rightarrow}$: se $f$ è continua, supponiamo per assurdo che esista $U \subset \mathbb{R}^m$ aperto tale che $f^{-1}(U)$ non è aperto in $A$. Allora $\exists z \in f^{-1}(U)$ tale che $\forall k \geq 1, B(\underline{z}, \frac{1}{k}) \not\subset f^{-1}(U)$. Ma questo implica che $\exists \underline{x_k} \in A \cap B(\underline{z}, \frac{1}{k})$ tale che $\underline{x_k} \not\in f^{-1}(U)$. Ma allora 
$$
|\underline{x_k} - \underline{w}| < \frac{1}{k} \to 0 \implies f(\underline{x_k}) \to f(\underline{z})
$$
ma allora $f(x_k) \in U^c$. \ \footnote{se appartenesse a $U$ allora $\underline{x_k} \in f^{-1}(U)$ il che contraddice come abbiamo definito la successione} \\ Il fatto che $f(z) \not\in U$ contraddice l'ipotesi iniziale. \\
$\boxed{\Leftarrow}$: sappiamo che la preimmagine di aperti è aperta in $A$ allora sia $\underline{x_k} \to \underline{z}$ con $\underline{x_k} \in A$ e $\underline{z} \in A$: se supponiamo, per assurdo, $f(\underline{x_k}) \not\to f(\underline{z})$ allora $\exists \alpha: \mathbb{N} \to \mathbb{N}$ tale che $\alpha(k) \stackrel{k \to +\infty}{\to} +\infty$ ed esiste $\sigma > 0: |f(\underline{x}_{\alpha(k)} - f(\underline{z})| \geq \sigma$. D'altra parte $f^{-1}(B(f(z), \sigma)$ è aperto in $A$ (siccome preimmagine di una balla, che è aperta), dunque esiste $\delta > 0$ tale che $A \cap B(z, \delta) \subset f^{-1}(B(f(z), \sigma)$ ed esiste $k_0 \geq 1$ tale che $x_k \in A \cap B(z, \delta) \subseteq f^{-1}(B(f(z), \sigma) \, \forall k \geq k_0 \implies |f(\underline{x_k}) - f(\underline{z})| \leq \sigma$ e questo contraddice la precedente disequazione $|f(\underline{x}_{\alpha(k)}-f(\underline{z})| \geq \sigma$ per $k$ sufficientemente grande.
\end{proof}
\begin{theorem}[teorema della permanenza del segno]
Sia $f: A \to \mathbb{R}$ continua in $z \in A$. Se $f(z) > 0$ allora $\exists \delta > 0$ tale che $\forall x \in A \cap B(z, \delta), f(x) > 0$
\end{theorem}
\begin{proof}
Se neghiamo la tesi allora otteniamo che $\forall \delta > 0, \exists x \in A \cap B(z, \delta): f(x) < 0$. Ma allora deduciamo che $\exists \underline{x_k} \in A \cap B(z, \frac{1}{k}): f(x_k) <0$, ma $\underline{x_k} \to z$ e $f(x_k) \to f(z) \leq 0$ che rappresenta una contraddizione.
\end{proof}
\begin{theorem}[teorema C2]
Siano $A \subset \mathbb{R}^n$, $f, g: A \to \mathbb{R}^m$ e $h: B \to \mathbb{R}^k$ con $f(A) \subset B \subset \mathbb{R}^m$. Se $f$ e $g$ sono continue in $z \in A$ e $h$ è continua in $f(z) \in B$ allora valgono le seguenti proprietà:
\begin{enumerate}[label=\protect\circled{\arabic*}]
	\item $\forall \lambda, \mu \in \mathbb{R}$ abbiamo che $\lambda f + \mu g$ è continua in $z \in A$;
	\item Se $m = 1$, allora $fg$ è continua in $z$;
	\item Se $m=1$ e $g(z) \neq 0$ allora $\exists \delta > 0$ per cui $f/g$ è ben definita su $A \cap B(z, \delta)$ ed è continua in $z$;
	 \item la composizione $h \circ f$ è continua in $z$
\end{enumerate}
\end{theorem}
\begin{proof}
Per mostrare la $\circled{1}$ osserviamo che, prendendo $x_k \to z$, abbiamo che
$$
\lambda f(x_k) + \mu g(x_k) \to \lambda f(z) + \mu g(z)
$$
Per mostrare la $\circled{2}$ osserviamo che
$$
x_k \to z \implies f(x_k)g(x_k) \to f(z)g(z)
$$
Per mostrare la $\circled{3}$ si osserva che possiamo supporre che $g(z) > 0$ senza perdita di generalità da cui segue che
$$
\exists \delta > 0: \forall x \in A \cap B(z, \delta), g(x) > 0 \implies \frac{f}{g}: B(z, \delta) \to \mathbb{R}
$$
che ci permette di concludere che $\frac{f}{g}$ sia ben definita e $\lim\limits_{k \to +\infty} \frac{f(x_k)}{g(x_k)} = \frac{f(z)}{g(z)}$ con $x_k \to z$. \\
La $\circled{4}$ segue banalmente dal fatto che presa $x_k \in A, x_k \to z$ allora
$$
\lim_{k \to +\infty} h(f(x_k)) = h(\lim_{k \to +\infty} f(x_k)) = h(f(z))
$$
per la continuità di $h$ in $f(z) \in f(A)$ e di $f$ in $z \in A$.
\end{proof}
\begin{prop}
Sia $f: A \to \mathbb{R}^m$, dove $f(x) = (f_1(x), f_2(x), \ldots f_m(x))$ e $f_j(x): A \to \mathbb{R} \, \forall j \in \{1, \ldots m\}$ con $A \subset \mathbb{R}^n$. Allora abbiamo che
	$$f \text{ è continua in } p \in A \iff f_j \text{ è continua in p} \in A \, \forall j \in \{1, \ldots m\}$$
\end{prop}
\begin{proof}
Basta osservare che $f(x_k) \to f(z) \iff f_j(x_k) \to f_j(z) \, \forall j \in \{1, \ldots m \}$
\end{proof}
\begin{remark}
$f$ è continua in $A \iff f_j$ è continua in $A \, \forall j \in \{1, \ldots m \}$.
\end{remark}
\noindent A questo punto ricordiamo la definizione di massimo e minimo locale o globale
\begin{definition}[massimo globale]
Sia $A \subseteq \mathbb{R}^n$ e sia $f: A \to \mathbb{R}$. Diremo che $z \in A$ è un punto di massimo assoluto, o globale, su $A$ se$$\forall x \in A, f(z) \geq f(x)$$.
\end{definition}
\noindent La definizione di minimo globale è analoga (basta sostituire il $\geq$ con $\leq$). \\
Diamo adesso la definizione di massimo e minimo locale:
\begin{definition}[massimo locale]
Sia $A \subseteq \mathbb{R}^n$ e sia $f: A \to \mathbb{R}$. Diremo che $z \in A$ è un punto di massimo locale su $A$ se
$$
\exists \delta > 0: \forall x \in A \cap B(z, \delta), f(z) \geq f(x)
$$
\end{definition}
\noindent analogamente si ottiene la definizione di minimo locale. \\ Data una funzione $f:A \to \mathbb{R}$ con $A \subset \mathbb{R}^n$, introdurremo la seguente notazione 
\begin{align*}
&\max_A{f} & &\min_A{f} 
\end{align*}
per indicare, rispettivamente, il massimo e il minimo assoluto. \\
Enunciamo il seguente teorema, valido in $\mathbb{R}^n$, che garantisce l'esistenza del massimo e del minimo di una funzione continua quando mappa un insieme compatto ad un altro. \\
\begin{theorem}[teorema di Weierstrass]
Sia $A \subset \mathbb{R}^n$ un insieme compatto e sia $f: A \to \mathbb{R}$ una funzione continua su $A$. Allora $\exists \max\limits_A{f}, \min\limits_A{f}$. 
\end{theorem}
\begin{proof}
Per caratterizzazione del $\sup_A{f}$ sappiamo che esiste una successione $\lim y_k$ dove $y_k \in f(A) \, \forall k \geq 1$ che vi ci tende. Siccome $y_k \in f(A)$ allora $\exists x_k \in A: f(x_k) = y_k \, \forall k \geq 1$. Ma allora noi sappiamo che, per al compattezza di $A$, che esiste una sottosuccessione $x_{\alpha (k)}$ tale che $x_{\alpha(k)} \to z \in A$, da cui si deduce che
$$
f(z) = \lim_{k \to +\infty} f(x_{\alpha(k)}) = \lim_{k \to +\infty} y_{\alpha(k)} = \sup_A{f}
$$
dove la prima uguaglianza segue dalla continuità di $f$ e la terza uguaglianza segue dal fatto che $y_{\alpha(k)}$ è una sottosuccessione estratta di $y_k$, ma siccome $y_k$ converge al $\sup_A{f}$ allora anche $y_{\alpha(k)}$ vi convergerà.
\end{proof}
\noindent Tramite il seguente teorema viene mostrato che la continuità preserva anche la connessione per archi di un insieme:
\begin{theorem}[teorema C4]
	Sia $f: C \to \mathbb{R}^m$ continua e sia $C$ connesso per archi. Allora $f(C) \subset R^m$ è connesso per archi
\end{theorem}
\begin{proof}
Siano $y, z \in f(C) \implies \exists x, u \in C$ tali che $f(x) = y$ e $f(u) = z$, pertanto $\exists \gamma : [0,1] \to C$ grazie alla connettività per archi di $C$ per cui $\gamma(0) = x$ e $\gamma(1) = u$. Ma allora la curva $f \circ \gamma: [0,1] \to f(C)$ è continua (siccome composizione di funzioni continue) e $(f \circ \gamma)(0) = y$ e $(f \circ \gamma)(1) = z$, dunque $f(C)$ è connesso per archi.
\end{proof}
\begin{cor}
$f: C \to \mathbb{R}$ continua, $C$ è connesso per archi $\implies f(C) \subset \mathbb{R}$ è un intervallo. 
\end{cor}
\begin{proof}
Dal precedente teorema segue, naturalmente, che $f(C)$ è connesso per archi. Mostriamo che è un intervallo: dati $t, s \in f(C), \exists \Gamma: [0,1] \to f(C) \subset \mathbb{R}$ continua tale che $\Gamma(0) = t$ e $\Gamma(1) = s$. Dunque $[t,s] \subseteq \Gamma([0, 1]) \subseteq f(C) \implies f(C)$ è un intervallo
\end{proof}
\noindent Da questo corollario è anche possibile anche dimostrare la seguente proposizione
\begin{prop}
Sia $A$ connesso per archi e sia $f: A \to \mathbb{R}$ continua. Allora
\begin{enumerate}[label=\protect\circled{\arabic*}]
	\item Se $\exists \max\limits_{A} f, \min\limits_{B} f$ allora $f(A) = [\min\limits_{A} f, \max\limits_{A} f]$
	\item Se $\not\exists \max\limits_{A} f, \exists \min\limits_{B} f$ allora $f(A)=[\min\limits_{A} f, \sup\limits_{A} f)$
\end{enumerate}
\end{prop}
\begin{exercise}
Mostrare la proposizione precedente
\end{exercise}
\begin{proof}
Per il corollario sappiamo che $f(A)$ è un intervallo. \\
Se il minimo e il massimo della funzione sono ben definiti, allora l'immagine è banalmente contenuta fra il minimo e il massimo la cui esistenza è garantita dal teorema di Weierstrass. \\
Per quanto riguarda il secondo caso, possiamo supporre, senza perdere di generalità, che $\not\exists \max\limits_{A} f$. Se supponiamo per assurdo che $f(A)$ non sia un intervallo allora esisterebbe un $x \in f(A)$ tale che $x$ non è compreso fra $\min_{A} f$ e $\sup_{A} f$: allora, per un $\delta$ sufficientemente piccolo, posso trovare un intorno di $x$ tale che $\sup\limits_A f < x - \varepsilon < x \in f(A)$ il che va contro la definizione di estremo superiore di $f$ come minimo dei maggioranti di $f(A)$.
\end{proof}
\section{Accenni ai limiti in più variabili}
\begin{definition}[definizione di limite]
Sia data $f: A \to \mathbb{R}^k, A \subseteq \mathbb{R}^n$ e sia $x_0 \in \partial A$. Allora diciamo che $f$ tende a $v \in \mathbb{R}^k$ per $x$ che tende a $x_0$ e scriviamo che
$$
\lim_{x \to x_0} f(x) = v \iff \forall \varepsilon > 0 \exists \delta > 0 : \forall x \in B(x_0, \delta) \cap A \setminus \{ x_0 \}, |f(x) - v| < \varepsilon
$$
\end{definition}
\noindent Diamo adesso la definizione di limite di $f(x)$ che tende all'infinito:
\begin{definition}
Sia data $f: A \to \mathbb{R}^k, A \subseteq \mathbb{R}^n$ illimitato. Diremo che $f$ tende all'infinito per $x$ che tende all'infinito e scriveremo che
$$
\lim_{|x| \to \infty} f(x) = \infty \iff \forall M > 0, \exists N > 0: \forall x \in A, |x| > N, |f(x)| > M
$$
\end{definition}
\begin{remark}
In generale tutte queste definizioni che stiamo dando funzionano in qualunque spazio metrico $(X,d)$ sostituendo al posto dell'usuale distanza euclidea definita in $\mathbb{R}^n$ la distanza $d$.
\end{remark}
\noindent Potremo scrivere anche le definizioni per gli altri due casi che ci restano, ma è banale e lo lasciamo come semplice esercizio teorico al lettore. \\
\begin{theorem}[caratterizzazione del limite per successioni]
Sia $f: A \to \mathbb{R}^m, A \subseteq \mathbb{R}^n, q \in \partial A, v \in \mathbb{R}^m$. Abbiamo che
$$
\lim_{x \to q} f(x) = v \iff (\forall \{ p_k \} \subseteq A \setminus \{ q \}, p_k \to q) \implies \lim_{k \to +\infty} f(p_k) = v
$$
\end{theorem}
\begin{proof} \hspace{1em} \newline
$\boxed{\Rightarrow}$: si osserva che, presa una qualunque successione $p_k$ che soddisfa le ipotesi della proposizione sulla destra, avremo che il limite $\lim_{k \to +\infty} f(p_k)$ è un limite dove avviene la composizione della successione $p_k$ con la funzione $f$ e sono verificate tutte le ipotesi riguardo al cambio di variabile nel limite, dunque $\lim_{k \to +\infty} f(p_k) = \lim_{x \to q} f(x) = v$. \\
$\boxed{\Leftarrow}$: osserviamo che se procedessimo per assurdo, negando la definizione di limite, avremo che $\exists \varepsilon: \forall \delta > 0, \exists x \in B(q, \delta) \cap A \setminus \{ q \}: |f(x) - v| \geq \varepsilon$. Ma allora, restringendosi a degli intorni di $q$ via via sempre più piccoli, come $V_k = (q-\frac{1}{k}, q + \frac{1}{k})$ esisterebbe $x_k \in A \cap V_k : |f(x_k) - v| > \varepsilon$. Ma questo è un assurdo, siccome $x_k \to q$ e ma $f(x_k) \not\to v$ contraddicendo la nostra ipotesi iniziale.
\end{proof}
\noindent Da questo teorema importantissimo segue questo semplice corollario (la cui dimostrazione è omessa siccome è banale)
\begin{theorem}[teorema del confronto]
Siano $h, g, f: A \to \mathbb{R}, x_0 \in \partial A, l \in \mathbb{R}$ e supponiamo che $\exists \delta > 0$ con
\begin{align*}
h(x) \leq f(x) \leq g(x) \, \, \forall x \in A \cap B(x_0, \delta) \setminus \{ x_0 \}
\end{align*} \\
Se 
$$\lim\limits_{x \to x_0} h(x) = \lim\limits_{x \to x_0} g(x) = l \implies \lim_{x \to x_0} f(x) = l$$
\end{theorem}
\begin{proof}
La dimostrazione è analoga a quella di Analisi 1 con qualche accorgimento sugli insiemi di definizione
\end{proof}
\noindent Prima di procedere sulla parte del calcolo differenziale, diamo una breve introduzione ai simboli di Landau (i cosiddetti o-piccoli e O-grandi) che semplificano notevolmente il calcolo in più variabili:
\begin{definition}[o-piccolo]
Siano $x_0 \in \partial A, A \subseteq \mathbb{R}^n$ e $f: A \to \mathbb{R}^m$ e $g: A \to \mathbb{R}^k$. Diremo che $f$ è un o-piccolo di $g$ per $x$ che tende a $x_0$, in simboli
$$
f = o(g) \text{ per } x \to x_0
$$
se $$\forall \varepsilon > 0 \, \exists \delta > 0 : |f(x)| \leq \varepsilon |g(x)| \, \forall x \in A \cap B(x_0, \delta) \setminus \{ x_0 \}$$
\end{definition}
\begin{definition}[O-grande]
Siano $x_0 \in \partial A, A \subseteq \mathbb{R}^n$ e $f: A \to \mathbb{R}^m$ e $g: A \to \mathbb{R}^k$. Diremo che $f$ è un O-grande di $g$ per $x$ che tende a $x_0$, in simboli
$$
f = O(g) \text{ per } x \to x_0
$$
se $$\exists M > 0, \delta > 0 : |f(x)| \leq M|g(x)| \, \forall x \in A \cap B(x_0, \delta) \setminus \{ x_0 \} $$
\end{definition}
\begin{exercise}
Se $f$ è un o-piccolo di $g$ allora $f$ è O-grande di $g$
\end{exercise}
\begin{proof}
Segue direttamente dalla definizione, siccome se $f = o(g)$ allora $\forall \varepsilon > 0 \exists \delta > 0: |f(x)| \leq \varepsilon |g(x)| \forall x \in A \cap B(x_0, \delta) \setminus \{ x_0 \}$, dunque fissando $M = \varepsilon > 0$ allora sappiamo che esiste un $\delta(\varepsilon)$ per cui
$$
|f(x)| \leq M |g(x)| \forall x \in A \cap B(x_0, \delta) \setminus \{ x_0 \}
$$ 
dunque risulta che $f$ è un O-grande di $g$, ovvero la tesi.
\end{proof}
\begin{prop}
Siano $A \subseteq \mathbb{R}^n, x_0 \in  A$ e $f: A \to \mathbb{R}^m$ e $\alpha > 0$. Allora abbiamo
$$
f(x) = o(|x-x_0|^\alpha) \text{ per } \, x \to x_0 \iff \lim_{x \to x_0} \frac{f(x)}{|x-x_0|^\alpha} = 0
$$
\end{prop}
\begin{proof}
la dimostrazione è banale, siccome 
\begin{align*}
&\forall \varepsilon > 0 \exists \delta > 0: |f(x)| \leq \varepsilon |x-x_0|^\alpha \, \forall x \in A \cap B(x_0, \delta) \setminus \{ x_0 \} \iff \forall \varepsilon > 0 \exists \delta > 0: \frac{|f(x)|}{|x-x_0|^\alpha} \leq \varepsilon \, \forall x \in A \cap B(x_0, \delta) \setminus \{ x_0 \} \\ &\iff \lim_{x \to x_0} \frac{f(x)}{|x-x_0|^\alpha} = 0
\end{align*}
\end{proof}
\begin{prop}
Siano $A \subseteq \mathbb{R}^n, x_0 \in A, f: A \to \mathbb{R}$ e $\alpha > 0$. Allora abbiamo che $$f(x) = O(|x-x_0|^\alpha) \text{ per } x \to x_0 \iff \exists M, \delta > 0: \frac{f(x)}{|x-x_0|^\alpha} \leq M \, \forall x \in A \cap B(x_0, \delta) \setminus \{ x_0 \}$$
\end{prop}
\begin{proof}
Analoga alla dimostrazione fatta sopra
\end{proof}
\noindent L'ultima proposizione si può anche riscrivere dicendo che se $f=O(|x-x_0|^\alpha)$ allora abbiamo necessariamente, per $x \to x_0$, che $\sup\limits_{A \cap B(x_0, \delta) \setminus \{ x_0 \}} \Bigg| \frac{f(x)}{|x-x_0|^{\alpha}} \Bigg| < +\infty$. \\
Osserviamo che
$$
\lim_{x \to x_0} f(x) = l \iff f(x) = l + o(1) \text{ per } x \to x_0
$$
e quindi potremmo (con tanta buona volontà) andare a riscrivere tutta la teoria appena fatta sui limiti in più variabili tramite i simboli di Landau. Ciò non verrà fatto, ma li useremo per introdurre, come avevo accennato, il calcolo differenziale.
\chapter{Calcolo differenziale}
In questo capitolo andremo a sviscerare il concetto di derivata quando abbiamo a che fare con funzioni vettoriali (sia nell'insieme di definizione che nell'insieme di arrivo) e come sia possibile ottenere massimi e minimi di una funzione introducendo dei concetti sofisticati come la \emph{matrice jacobiana} e la \emph{matrice hessiana}.
\section{Funzione differenziabile}
\begin{definition}[funzione differenziabile in un punto e matrice jacobiana]
Sia $\Omega \subseteq \mathbb{R}^n$ aperto e $x_0 \in \Omega$. Diremo che $f: \Omega \to \mathbb{R}^m$ è differenziabile in $x_0$ se $\exists L: \mathbb{R}^n \to \mathbb{R}^m$ lineare tale che
$$
\frac{f(x_0 + h)-f(x) - L(h)}{|h|} \stackrel{h \to 0}{\to} 0
$$
dove l'applicazione $L = df(x_0): \mathbb{R}^n \to \mathbb{R}^m$ è detto il differenziale di $f$ in $x_0$. La matrice $Df(x_0) \in \mathbb{R}^{m \times n}$ che rappresenta $df(x_0)$ rispetto alla base canonica è detta \textbf{matrice jacobiana} di $f$ in $x_0$.
\end{definition}
\begin{remark}
Naturalmente diremo che $f$ è differenziabile in $\Omega$ se è differenziabile $\forall x \in \Omega$.
\end{remark}
\noindent Come avevo anticipato alla fine dello scorso capitolo, possiamo rendere molto semplice la trattazione del calcolo differenziabile tramite i simboli di Landau. Mostriamo per esempio la seguente proposizione:
\begin{prop}
$f: \Omega \to \mathbb{R}^m$ con $\Omega \subseteq \mathbb{R}^n$ aperto. Diremo che $f$ è differenziabile in $x_0 \in \Omega \iff \exists L:\mathbb{R}^n \to \mathbb{R}^m$ lineare tale che $f(x_0 + h) = f(x_0) + L(h) + o(h)$ per $h \to 0 \iff $(cambio di variabile) $\exists L: \mathbb{R}^n \to \mathbb{R}^m$ lineare tale che $f(x) = f(x_0) + L(x-x_0) + o(x-x_0)$ per $x \to x_0$.
\end{prop}
\begin{proof}
Se $f$ è differenziabile in $x_0 \in \Omega$, sappiamo che $\exists L : \mathbb{R}^n \to \mathbb{R}^m$ tale che
$$
\frac{f(x_0 + h) - f(x_0) - L(h)}{|h|} \to 0 \implies f(x_0 + h) - f(x_0) - L(h) = o(|h|) \text{ per } h \to 0
$$
che è equivalente (per cambiamento di variabile) a:
$$
f(x) = f(x_0) + L(x-x_0) + o(|x-x_0|) \text{ per } x \to x_0
$$
Il viceversa è analogo, siccome se
$$
f(x_0 + h) - f(x_0) - L(h) = o(|h|) \implies \frac{f(x_0 + h) - f(x_0) - L(h)}{|h|} \to 0 \implies f \text{ è differenziabile in } x_0
$$
\end{proof}
\noindent Ma in sostanza che cosa vuol dire essere differenziabili? Come forse alcuni avranno potuto capire dalla definizione, essere differenziabili vuol dire che è possibile approssimare la funzione, in quel punto $x_0 \in \Omega$ in cui è differenziabile, ad una applicazione lineare affine del tipo $$a(x) = f(x_0) + df(x_0)(x-x_0)$$ che sarebbe, in maniera impropria, ciò che, in gergo da fisici, diciamo essere un'approssimazione del primo ordine. \\
Dopo questa definizione siamo persino pronti a definire il concetto di derivata direzionale, nozione strettamente connessa a quella di differenziabilità.
\begin{definition}[derivata direzionale]
Sia $\Omega \subseteq \mathbb{R}^n$ aperto, $x_0 \in \Omega$, $f: \Omega \to \mathbb{R}^n$ e sia $v \in \mathbb{R}^n \setminus \{ \underline{0} \}$. Definiamo la derivata direzionale rispetto a $v$ il limite (se esiste!)
$$
\partial_v f(x_0) = \lim_{t \to 0} \frac{f(x_0 + tv) - f(x_0)}{t}
$$
\end{definition}
\noindent Nel caso in cui $v = e_i$ la derivata direzionale è detta derivata parziale rispetto a $x_i$.
\begin{remark}
La derivata parziale rispetto a $x_i$ è, di fatto, la derivata di una funzione di una sola variabile. Pertanto queste posso essere svolte tenendo "costanti" le altre variabili $x_j$ con $j \neq i$ e calcolare la derivata come usualmente si faceva ad Analisi 1
\end{remark}
\noindent Ricollegandoci all'osservazione fatta qua sopra, infatti, possiamo osservare che:
$$
\partial_{e_1} f(x_0) = \lim_{t \to 0} \frac{f(x_0 + te_1) - f(x_0)}{t} = \lim_{t \to 0} \frac{f(x_1 + t, x_2, \ldots) - f(x_1, x_2, \ldots )}{t}
$$
Siamo adesso pronti ad enunciare il seguente risultato, che ci dà qualche informazione su come sia il differenziale:
\begin{theorem}[teorema DF1]
Se $f: \Omega \to \mathbb{R}$ è differenziabile in $x_0$ allora $\exists \partial_v f \, \forall v \in \mathbb{R}^n \setminus \{ \underline{0} \}$ e vale che
$$
df(x_0) = \partial_v f(x_0)
$$
\end{theorem}
\begin{proof}
Siccome $f$ è differenziabile in $x_0 \in \Omega$ allora $\exists L: \mathbb{R}^n \to \mathbb{R}$ tale che $f(x_0 + tv) = f(x_0) + L(tv) + o(|tv|)$ dove $|tv| \to 0$ per $v \in \mathbb{R}^n \setminus \{ \underline{0} \}$. Ma questo allora implica che
\begin{align*}
&\frac{f(x_0 + tv) - f(x_0) - L(|tv|)}{t} \to 0 \implies \frac{f(x_0 + tv) - f(x_0) - tL(v)}{t} \to 0 \implies \lim_{t \to 0} (\frac{f(x_0 + tv) - f(x_0)}{t} - L(v)) \to 0 \implies \\ &\implies \exists \partial_v f(x_0) = L(v) = df(x_0)(v)
\end{align*}
\end{proof}
\begin{theorem}[rappresentazione e unicità del differenziale]
Se $f: \Omega \to \mathbb{R}$ è differenziabile in $x_0 \in \Omega$  allora il differenziabile di $f$ nel punto $x_0 \in \Omega$ è unico e
$$
df(x_0)(\xi_1, \ldots \xi_n) = \sum_{j=1}^n \xi_j \frac{\partial f}{\partial x_j}(x_0)
$$
\end{theorem}
\begin{proof}
Preso $v \in \mathbb{R}^n \setminus \{ \underline{0} \}$, allora 
$$
v = \sum \xi_i e_i \simeq (\xi_1, \ldots, \xi_n) \implies df(x_0)(v) = \sum_{j = 1}^n \xi_j df(x_0)(e_j) = \sum_{j=1}^n \xi_j \partial_{e_j} f(x_0)
$$
dove abbiamo solamente sfruttato la linearità della funzione differenziabile e il fatto che $df(x_0)(e_j) = \frac{\partial f}{\partial x_j}(x_0)$. L'unicità del differenziabile deriva dal fatto che se supponiamo per assurdo che esista $L' \neq df(x_0)$ lineare, tali che $L(v) = df(x_0)(v) \, \forall v \in \mathbb{R}^n \setminus \{ \underline{0} \}$ allora preso un $v \in \mathbb{R}^n \setminus \{ \underline{0} \}$ avremo che:
$$
L(v) = \sum_j^n \xi_j L(e_j) = \sum_j^n \xi_j df(x_0)(e_j) = df(x_0)(v)
$$
dove tutte queste uguaglianze sono ottenute sfruttando la linearità di $L$ e $df(x_0)$. L'unica possibilità (siccome questa è una relazione valida per ogni vettore) è che $L(e_j) = df(x_0)(e_j) \, \forall j$ ma allora queste due applicazioni lineari coincidono, giungendo ad un assurdo. Quindi il differenziale $df(x_0)$ è unico.
\end{proof}
\noindent Possiamo dunque definire il concetto di gradiente:
\begin{definition}[gradiente di una funzione]
Sia $f: \Omega \to \mathbb{R}$ con $\Omega \subseteq \mathbb{R}^n$, $x_0 \in \Omega$ e $f$ differenziabile in $x_0$. Allora definiamo il gradiente $$\nabla{f(x_0)} = (\frac{\partial f}{\partial x_1}(x_0), \ldots, \frac{\partial f}{\partial x_n}(x_0)) = \sum_{i=1}^n \frac{\partial f}{\partial x_i}f(x_0)e_i \in \mathbb{R}^n$$
\end{definition}
\begin{remark}
Il gradiente di $f$ nel punto $x_0$ è di fatto il vettore che ha come componenti le derivate parziali di $f$
\end{remark}
\noindent Dalla formula precedente è possibile ricavare che
$$
df(x_0)(v) = \innerprod{\nabla{f(x_0)}}{v}
$$
\begin{remark}
Da qua possiamo vedere il grandissimo problema di definire il gradiente: per definire le derivate parziali, di fatto, abbiamo solamente fatto riferimento alla natura da "spazio metrico" di $\mathbb{R}^n$, quindi in generale potremmo andare a definire la derivata su una classe maggiore di insiemi rispetto che a $\mathbb{R}$ e $\mathbb{C}$. Per definire il gradiente, però, abbiamo richiesto il prodotto scalare, che è una condizione decisamente più forte che alla semplice distanza.
\end{remark}
\end{document}